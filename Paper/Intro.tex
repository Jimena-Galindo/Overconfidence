% Options for packages loaded elsewhere
\PassOptionsToPackage{unicode}{hyperref}
\PassOptionsToPackage{hyphens}{url}
%
\documentclass[
  12pt,
]{article}
\usepackage{amsmath,amssymb}
\usepackage{iftex}
\ifPDFTeX
  \usepackage[T1]{fontenc}
  \usepackage[utf8]{inputenc}
  \usepackage{textcomp} % provide euro and other symbols
\else % if luatex or xetex
  \usepackage{unicode-math} % this also loads fontspec
  \defaultfontfeatures{Scale=MatchLowercase}
  \defaultfontfeatures[\rmfamily]{Ligatures=TeX,Scale=1}
\fi
\usepackage{lmodern}
\ifPDFTeX\else
  % xetex/luatex font selection
\fi
% Use upquote if available, for straight quotes in verbatim environments
\IfFileExists{upquote.sty}{\usepackage{upquote}}{}
\IfFileExists{microtype.sty}{% use microtype if available
  \usepackage[]{microtype}
  \UseMicrotypeSet[protrusion]{basicmath} % disable protrusion for tt fonts
}{}
\makeatletter
\@ifundefined{KOMAClassName}{% if non-KOMA class
  \IfFileExists{parskip.sty}{%
    \usepackage{parskip}
  }{% else
    \setlength{\parindent}{0pt}
    \setlength{\parskip}{6pt plus 2pt minus 1pt}}
}{% if KOMA class
  \KOMAoptions{parskip=half}}
\makeatother
\usepackage{xcolor}
\usepackage[margin=2.5cm]{geometry}
\setlength{\emergencystretch}{3em} % prevent overfull lines
\providecommand{\tightlist}{%
  \setlength{\itemsep}{0pt}\setlength{\parskip}{0pt}}
\setcounter{secnumdepth}{5}

\setlength{\parindent}{25pt}
\setlength{\parskip}{0pt}
\setlength{\skip\footins}{0.25cm} % margin before footnotes
\definecolor{myblue}{RGB}{31, 61, 122}
\definecolor{mypink}{RGB}{204, 0, 82}
\newtheorem{assumption}{Assumption}
\newtheorem{proposition}{Proposition} % Use the 'proposition' counter for numbering
\usepackage{rotating, graphicx}

\newcommand{\EE}[1]{\mathbb{E}\left[#1\right]}
\newcommand{\PP}[1]{\mathbb{P}\left(#1\right)}
\newcommand{\CP}[2]{\mathbb{P}\left(#1 \,| \, #2 \right)}
\newcommand{\CE}[2]{\mathbb{E}\left[#1\,|\,#2 \right]}
\usepackage[bottom]{footmisc}
\usepackage[doublespacing]{setspace}
\usepackage[normal]{caption}
\usepackage{dsfont}
\usepackage{booktabs}
\usepackage{makecell}
\usepackage{hyperref}
\ifLuaTeX
  \usepackage{selnolig}  % disable illegal ligatures
\fi
\usepackage[]{natbib}
\bibliographystyle{plainnat}
\IfFileExists{bookmark.sty}{\usepackage{bookmark}}{\usepackage{hyperref}}
\IfFileExists{xurl.sty}{\usepackage{xurl}}{} % add URL line breaks if available
\urlstyle{same}
\hypersetup{
  pdftitle={Learning with Misspecified Models: the case of overestimation},
  pdfauthor={Jimena Galindo},
  hidelinks,
  pdfcreator={LaTeX via pandoc}}

\title{Learning with Misspecified Models: the case of overestimation}
\author{Jimena Galindo}
\date{October 24, 2023}

\begin{document}
\maketitle
\begin{abstract}
I design a framework and a laboratory experiment that allows for the
comparison of multiple theories of misspecified learning. I focus on a
framework with endogenous information and a data-generating process
ruled by two fundamentals: an ego-relevant parameter and a state. Within
this framework I study three forces that can lead to misspecified
beliefs: initial misspecifications, learning traps and biased updating.
I find that biased updating is the main driver of misspecified beliefs
in the lab. In addition, I vary the degree of ego-relevance of the
parameter by introducing a stereotype treatment. The data is consistent
with biased updating in both cases but for potentially different
reasons: when learning about themselves, subjects attribute successes to
their own ability and failures to luck. Instead, in the stereotype
treatment, they compensate for initial negative biases by
over-attributing positive signals to the ability of others. This
translates into similar observed choices but different dynamics in
beliefs.
\end{abstract}

\newpage

\hypertarget{introduction}{%
\section{Introduction}\label{introduction}}

A growing body of literature in economics explores how people develop
incorrect beliefs about fundamentals. Most of this research centers on
scenarios where agents passively observe the world, and incorrectly
integrate the received information into their
beliefs.\footnote{See \citet{benjamin2019} for a review of the literature on errors of 
probabilistic thinking.} However, many real-world situations cast agents
as active participants in the generation of information. In these cases,
the information they observe is influenced by their actions and
subsequent behavior is in turn determined by how they incorporate the
information into their beliefs.

As an example consider a student who needs to decide how much effort to
put into studying for an exam. Their decision will depend on two
factors: their belief about their intrinsic ability, and their belief
about how difficult the exam will be. The outcome they observe will be
affected by how much they decide to study. Imagine that the student puts
in a moderate amount of effort and gets a surprisingly good grade. Did
they get a good grade because they are smarter than they thought? Or
because the exam was easier than they had anticipated? Their future exam
preparation strategy will depend fundamentally on which line of
reasoning they take. This feedback loop is referred to as an
\emph{endogenous information 
process} and is at the center of the forces I study.

To understand what the main forces at play are, I compare a set of
theories that model learning in settings with endogenous information,
and which can rationalize the persistence of misspecified beliefs. I
develop a unifying framework that nests multiple theories of learning
and generates testable predictions for each of them. Then I cast this
framework in a laboratory experiment and test the predictions to
identify which of the theories are consistent with the behavior I
observe in the lab. The experiment features an agent who needs to learn
two parameters: one that pertains their own characteristics (an
\emph{ego-relevant} parameter), and an exogenously determined state---in
the context of the student preparing for exams, the two parameters would
be their intrinsic ability and the difficulty of the exam.

Misspecified beliefs about an ego-relevant parameter are often referred
to as overconfidence (or underconfidence) and have been documented by
behavioral scientists\footnote{\citet{kelley1980} 
provides a review of the psychology literature.} and
economists\footnote{See \citet{benjamin2019} 
for a review of the literature in economics.} in a variety of settings.
\citet{Oster2013} find that subjects who are at high risk of having
Huntington's disease overestimate their probability of being healthy and
make retirement decisions as if they were healthy. \citet{Hoffman2020}
show that workers overestimate the quality of their match to their
current employment and are unlikely to look for other opportunities.
\citet{Camerer1999} find that entrepreneurs are overconfident about the
quality of their enterprise, which leads to excessive entry and early
exit from markets.

In all of these examples, holding an incorrect belief about a
fundamental leads to suboptimal choices with potentially high costs. In
spite of the abundance of evidence, the scope of the existing research
is limited in terms of the frameworks it considers. Most of the
experimental evidence documenting the bias is collected in settings
where subject are passive learners.\footnote{
\citet{Gotte2022} and \citet{Ozyilmaz2022} are exceptions that study settings with endogenous information processes.}
They observe a noisy signal and report their beliefs over subsequent
rounds.\footnote{\citet{Bracha2012} and 
\citet{Mobius2022} are some examples.} Although these studies provide
important insights, they are not flexible enough to incorporate the
richer theories that have been proposed more recently. In particular,
they do not allow the study of endogenous learning or for learning about
multiple parameters at once.\footnote{\citet{Coutts2020} 
studies an environment with an ego-relevant parameter and an exogenous state but does not 
incorporate the endogenous information process.}

In my experiment, I move away from the standard framework of passive
learning to analyze a richer set of learning mechanisms. In particular,
the interaction between the two parameters together with an endogenous
information process gives rise to three possible mechanisms that allow
for the persistence of incorrect beliefs: the presence of learning
traps, incorrect initial beliefs, and misattribution bias. The theories
that I consider incorporate different combinations of these mechanisms.

When the setting features learning traps, even an agent who incorporates
all information correctly, may fall into learning traps as outlined by
\citet{Hestermann2021}. These traps are characterized by a combination
of an incorrect belief and an optimal action which produce information
that confirms the incorrect belief. Once an agent falls into a trap, the
belief will be stable and even with a correctly specified model of
learning, they will not be able to abandon their misspecified beliefs.
If, the agent is dogmatic about their initial belief,
\citet{Heidhues2018} show that they will inevitably fall into a trap and
thus will be able to rationalize and sustain their initial
misspecification
belief.\footnote{\citet{Gotte2022} study the case of agents with dogmatic initial beliefs in a 
laboratory experiment.}

\citet{Ba2023} moves away from dogmatism and endows the agents with a
mechanism through which they can abandon incorrect beliefs; this allows
them to avoid falling into learning traps. To do so agents perform
Bayesian hypothesis tests that evaluate which is the more likely
parameter out of two possibilities. By doing this, she characterizes the
set of situations in which, even agents who consider alternative
paradigms, may become trapped.\footnote{A 
similar mechanism is proposed by \citet{Schwarstein2021} in a setting with persuasion.}

Lastly, misattribution bias is the more classical explanation and has
been widely studied in behavioral
science.\footnote{See \citet{kelley1980} for a review.} Agents who
suffer from misattribution bias will attribute successes to their own
ability---the ego-relevant parameter---and failures to bad luck---the
state. Under this model of learning, even an agent who initially has a
correct initial belief may become overconfident if they observe a
sequence of successes. In this case, the main driver of the bias is not
an initial misspecification or the presence of learning traps, it is the
updating procedure
itself.\footnote{A more general framework that can be used to model 
this bias has also been proposed by \citet{Brunnermeier2005} and empirically studied by \citet{Bracha2012}.}

These theories provide the main building blocks for a simplified
framework that can be directly implemented in a laboratory experiment.
In the experiment subjects make choices and receive feedback that
depends on their own ability, an exogenous parameter and the choice they
made. I track their choices as well as their beliefs about their own
ability. The goal is to identify which of the 3 forces---the presence of
learning traps, misspecified initial beliefs, or misatribution
bias---better explains the observed behavior. To determine the fit of
the models, I compare the behavior predicted by each theory to the
benchmark given by the fully Bayesian updating procedure.

I also study whether the learning mechanism is inherently linked to the
ego-relevance of the parameters or if it is a more general phenomenon. I
vary the degree of ego-relevance by introducing a treatment in which
subjects learn about the ability of another participant. In this
treatment, the participants know only the gender and nationality of the
other, and thus can induce stereotypes---a different type of
misspecification.\\
If correct learning about the parameters happens at higher rates in the
stereotype treatment, it would suggest that the bias is intrinsically
linked to the ego-relevance of the parameter. In contrast, if similar
biased behavior arises in both treatments, it is more likely that the
main driver of these types of misspecified beliefs is the updating
procedure itself or the endogenous information process.

Although some agents do fall into learning traps, I find that the
behavior of most subjects is better explained by misattribution bias:
good news are treated as signaling high ability, while bad news are
attributed to a low state. I also find that misattribution is no more
prevalent in the ego-relevant than in the stereotype condition. This
suggests that the main driver of the misspecification is the updating
procedure; however, the underlying mechanism by which the bias is
generated may be different in both treatments: While in the ego-relevant
condition subjects prefer to hold themselves in high esteem, in the
stereotype condition updating seems to be driven by some sort of bias
overcorrection---when subjects realize that they underestimated the
ability of another participant based on their gender and nationality,
they compensate by overestimating their ability.

Finally, I estimate the structural parameters of the models to study
model-heterogeneity in the sample. I find that even at an individual
level the behavior is better explained by a general model of
misattribution bias for most subjects. There is a smaller group of
subjects that can be better explained through dogmatic beliefs and
hypothesis testing and none of them behave in line with the fully
Bayesian benchmark.

In what follows I first discuss the theoretical framework and the
predictions of each of the theories. Then I introduce a unifying example
and my hypotheses. In section 4 I describe the experimental design and
in section 5 I present the data and the results. Section 6 outlines the
estimation of the parameters and the model fit analysis.

  \bibliography{references.bib}

\end{document}
