% Options for packages loaded elsewhere
\PassOptionsToPackage{unicode}{hyperref}
\PassOptionsToPackage{hyphens}{url}
%
\documentclass[
  12pt,
]{article}
\usepackage{amsmath,amssymb}
\usepackage{lmodern}
\usepackage{iftex}
\ifPDFTeX
  \usepackage[T1]{fontenc}
  \usepackage[utf8]{inputenc}
  \usepackage{textcomp} % provide euro and other symbols
\else % if luatex or xetex
  \usepackage{unicode-math}
  \defaultfontfeatures{Scale=MatchLowercase}
  \defaultfontfeatures[\rmfamily]{Ligatures=TeX,Scale=1}
\fi
% Use upquote if available, for straight quotes in verbatim environments
\IfFileExists{upquote.sty}{\usepackage{upquote}}{}
\IfFileExists{microtype.sty}{% use microtype if available
  \usepackage[]{microtype}
  \UseMicrotypeSet[protrusion]{basicmath} % disable protrusion for tt fonts
}{}
\makeatletter
\@ifundefined{KOMAClassName}{% if non-KOMA class
  \IfFileExists{parskip.sty}{%
    \usepackage{parskip}
  }{% else
    \setlength{\parindent}{0pt}
    \setlength{\parskip}{6pt plus 2pt minus 1pt}}
}{% if KOMA class
  \KOMAoptions{parskip=half}}
\makeatother
\usepackage{xcolor}
\IfFileExists{xurl.sty}{\usepackage{xurl}}{} % add URL line breaks if available
\IfFileExists{bookmark.sty}{\usepackage{bookmark}}{\usepackage{hyperref}}
\hypersetup{
  pdftitle={Learning with misspecified models: Overconfidence and Stereotypes},
  pdfauthor={Jimena Galindo},
  hidelinks,
  pdfcreator={LaTeX via pandoc}}
\urlstyle{same} % disable monospaced font for URLs
\usepackage[margin=2.5cm]{geometry}
\setlength{\emergencystretch}{3em} % prevent overfull lines
\providecommand{\tightlist}{%
  \setlength{\itemsep}{0pt}\setlength{\parskip}{0pt}}
\setcounter{secnumdepth}{5}

\setlength{\parindent}{25pt}
\setlength{\parskip}{0pt}
\setlength{\skip\footins}{0.25cm} % margin before footnotes
\definecolor{myblue}{RGB}{31, 61, 122}
\definecolor{mypink}{RGB}{204, 0, 82}
\newtheorem{assumption}{Assumption}
\newtheorem{proposition}{Proposition} % Use the 'proposition' counter for numbering

\newcommand{\EE}[1]{\mathbb{E}\left[#1\right]}
\newcommand{\PP}[1]{\mathbb{P}\left(#1\right)}
\newcommand{\CP}[2]{\mathbb{P}\left(#1 \,| \, #2 \right)}
\newcommand{\CE}[2]{\mathbb{E}\left[#1\,|\,#2 \right]}
\usepackage[bottom]{footmisc}
\usepackage[doublespacing]{setspace}
\usepackage[normal]{caption}
\usepackage{dsfont}
\usepackage{booktabs}
\usepackage{makecell}
\usepackage{hyperref}
\ifLuaTeX
  \usepackage{selnolig}  % disable illegal ligatures
\fi
\usepackage[]{natbib}
\bibliographystyle{plainnat}

\title{Learning with misspecified models: Overconfidence and
Stereotypes}
\author{Jimena Galindo}
\date{September 07, 2023}

\begin{document}
\maketitle
\begin{abstract}
TBW
\end{abstract}

\hypertarget{introduction}{%
\section{Introduction}\label{introduction}}

\hypertarget{related-literature}{%
\section{Related Literature}\label{related-literature}}

\hypertarget{framework-1}{%
\section{Framework 1}\label{framework-1}}

An agent is of type \(\theta \in \Theta\) and faces an unknown exogenous
state \(\omega\) drawn from some density \(f\) over \(\Omega\). The
agent knows the distribution of \(\omega\) but not its realized value.
His prior belief about his type is \(p_0(\theta)\) and his belief about
the state state \(p_0(\omega)\) coincides with the true distribution
\(f\). Let the agent's type be \(\theta^*\) and the realized state be
\(\omega^*\).

An agent has a \emph{misspecified} belief if the prior assigns
probability zero to their true type. Furthermore, the agent is
\emph{dogmatic} if he holds a degenerate belief that places probability
one on being a particular type, \(\hat{\theta}\). An agent can be
dogmatic and misspecified; that means that
\(\hat{\theta} \neq \theta^*\) and \(p_0(\hat{\theta}) = 1\).

The agent chooses an action \(a\in A\) and observes a noisy outcome
\(h\in H\). The outcome is a function of the agent's type, the state,
and the action. In particular
\(h = h(\theta^*, \omega^*, a) + \varepsilon\) with \(h(\cdot)\)
increasing in both \(\theta^*\) and \(\omega^*\), and such that
conditional on a pair of parameters \((\theta, \omega)\), there is a
unique optimal action. \(\varepsilon\sim N(0, \sigma)\) is noise in the
output.

After observing the outcome, the agent updates his beliefs about
\(\theta\) and \(\omega\) using some algorithm and moves on to the next
period. He repeats this process infinitely many times. I make the
simplifying assumption that the agent is myopic and chooses the action
that maximizes the payoff in each period. This assumption is symplifies
the analysis and plays a role on whether an agent who updates their
beliefs using bayes rule would learn the truth or not.

A key notion in this setting is that of a self-defeating
equilibrium\footnote{This notion is an adaptation of the Berk-Nash equilibrium in 
@Esponda2016 to this setting with only one agent}. A
\emph{self-defeating equilibrium} is a belief and action pair such that
the agent's belief about their type is misspecified and the outcome
generated by the action is consistent with the misspecified belief. The
average outcome under the true type and the true state equals the
average output the agent expects under the misspecified belief. In
addition, the agent's belief is said to be \emph{stable} when this
happens.

Within this framework, I consider two nested theories of belief
updating. The first one is a dogmatic modeler from \citet{Heidhues2018}.
The second one is a switcher, as in \citet{Ba2023}. The dogmatic modeler
can be seen as a switcher with an infinitely sticky initial belief and
this is the sense in which the two are nested. Both theories produce
different predictions with respect to the equilibrium outcome.

\hypertarget{the-dogmatic-modeler}{%
\subsection{The Dogmatic Modeler}\label{the-dogmatic-modeler}}

A dogmatic agent does not update their beliefs about \(\theta\); instead
he holds a degenerate belief that places probability one on being a
particular type, \(\hat{\theta}\), which is potentially not his true
type. In this case, no matter how much information he gathers against
being of type \(\hat{\theta}\), he will not update his beliefs. Any
discrepancies between the observed outcomes and his believed type are
incorporated using the Bayes rule to update their beliefs about
\(\omega\). This means the dogmatic agent will never learn their true
type if they startu off as being misspecified.

\citet{Heidhues2018} show that, under certain assumptions on the
per-period utility,\footnote{The assumptions are that 
$u$ is twice continuously differentiable with: (i)$u_{ee}<0$ and $u_e(\underline{e} \theta, \omega)>0>u_e(\Bar{e}, \theta, \omega)$, 
(ii) $u_{\theta}, u_{\omega}>0$ and (iii) $u_{e\theta}<0$ and $u_{e\omega}>0$. The direction of the derivatives is a normalization
and the results would hold even when the signs are reversed.} a dogmatic
modeler will inevitably fall into a self-defeating equilibrium. The
equilibrium will be such that the outcomes they observe reinforce their
belief on \(\omega\) in such a way that as \(t\to\infty\) the agent will
be sure that the state is some \(\omega^{'}\) consistent with their
believed type and the observed data. In other words, they will be in a
self-defeating equilibrium whith a stable belief that places probability
one on the incorrect parameters \((\hat{\theta}, \omega_{\infty})\)

The mechanism by which the dogmatic agent falls into the self-defeating
equilibrium is the following: Suppose the agent holds the misspecified
belief that they are type \(\hat{\theta}>\theta^*\). For any prior over
\(\omega\), the agent will be disappointed by the outcome. He expected a
gain of \(h(\hat{\theta}, \mathbb{E}(\omega), a)\) but instead observes
\(h(\hat{\theta}, \omega^*, a)\). There are two possible sources for the
disappointment, the first is that the realized state is lower than the
expected state. The second source is that the agent is of type
\(\theta^*\) and therefore, for all possible states, his gain will be
lower than what he expected. Because the agent is dogmatic, he will not
update his beliefs about \(\theta\) and as a consequence will attribute
the disappointment to the state being lower than expected. He will
continue to update in ths way until he converges to a belief about
\(\omega\) that is stable. Such a belief will explain the observed
utility perfectly and allow the agent to rationalize his dogmatic belief
about \(\theta\). Under the assumptions of \citet{Heidhues2018}, there
is a unique value of \(\omega\) at which the belief is stable, I will
refer to such value as \(\omega_\infty\). This mechanism is further
illustrated in Example 1.

\textbf{Example 1: } \emph{Set \(A = \Omega\) and \(H = [0, \infty)\)
and consider a student with intrinsic ability \(\theta^*\geq 0\) who
faces a grading procedure \(\omega^*\) that is unknown to them. However,
they know that a higher \(\omega^*\) is more likely to yield a higher
grade. In particular, assume the grade is given by
\((\theta^*+a)\omega^*\).}

\emph{The student must choose an effort level \(a\), which determines
their grade. For whatever the chosen effort level, the agent must pay a
cost \(c(a) = \frac{1}{2}a^2\). And he repeats this process for
infinitely many periods. Assume also that the student's prior is such
that \(mathbb{E}[\omega]= \omega^*\) and he is dogmatic about being of
type \(\hat{\theta}>\theta^*\).
\footnote{The example is illustrated for an overconfident agent but the results are symmetric for a digmatic agent who initially 
places probability one on some $\tilde{\theta}<\theta^*$.} Therefore,
the student's payoff in period \(t\) is given by }

\begin{equation}
u_t(a_t; \theta^*, \omega^*) = (\theta^*+a_t)\omega^* - \frac{1}{2}a^2 + \varepsilon_t
\end{equation}

\emph{Under this specification, the myopic optimal effort level is
\(a_t^* = \omega^*\). Nonetheless, because the agent does not know
\(\omega^*\), he will choose \(a_t = \mathbb{E}_t(\omega)\) where the
expectation is taken with respect to the agent's belief at the beginning
of period \(t\). If he does not revise his effort choice for \(k\)
periods, he will receive an average utility of
\((\theta^*+a_t^*)\omega^* - \frac{1}{2}a_t^{*2}\) but he was expecting
an average utility of
\((\hat{\theta}\theta+a_t^*)\omega^* - \frac{1}{2}a_t^{*2}\). In
response, he will apply bayes rule to update his beliefs about
\(\omega\) to get the posterior belief with
\(\mathbb{E}_{t+k}[\omega] = \frac{(\theta^{*} + \omega^{*})\omega^{*}}{\hat{\theta} + \omega^{*}}\)
which is lower than the initial belief. This will cause the agent to
choose a lower effort at \(t+k\). As a result, he will again receive an
average utlitity that is lower than what he expected which will cause
his belief to drift further down. This process will continue until the
average utility equals his expected utility under the dogmatic belief
that assigns probability 1 to \(\hat{\theta}\). At that point, the
stundent will have reached a self-defeating equilibrium and he will
continue to choose sub-optimal effort forever. }

Although the model of a dogmatic modeler is able to rationalize the
prevalence of overconfident (underconfident) beliefs, the assumption
that the agent has a degenerate belief and no mechanism through which he
can update such belief is very restrictive. An alternative approach is
proposed by \citet{Ba2023}. She proposes an extension of the dogmatic
agent who is able to jump from one dogmatic belief to another. By doing
so, the agent might end up being dogmatic and correctly speciefied.

\hypertarget{the-switcher}{%
\subsection{The Switcher}\label{the-switcher}}

An agent is a \emph{switcher} if they behave as a dogmatic, but is
willing to entertain the possibility that they are of a different type.
In particular, when they start off as a misspecified dogmatic, they are
willing to switch to a different dogmatic belief if the data is
convincing enough. Notice that their beliefs are still degenerate and
assign probability one to a particular type, and zero to all other
types. This means that a bayesian update on \(\theta\) would cause no
change in their beliefs. However, they are willing to entertain two such
beliefs and have a mechanism by which they decide which belief to adopt
at any period \(t\).

In order to abandon their initial dogmatic belief, the agent needs to
observe a sequence of outcomes that are sufficiently unlikely to have
happened if they were of the type they initially believed. They do so by
keeping track of the likelihood that each of the possible types
generated the data. If the likelihood ratio is sufficiently large, the
agent will switch to the alternative and behave as if they are dogmatic
about the new type.

In particular, for an agent that starts off with a dogmatic belief that
they are of type \(\hat{\theta}\) but is willing to consider the
alternative explanation that they are of type \(\tilde{\theta}\), the
agent will switch to the alternative if:

\[\frac{p[h^t|\tilde{\theta}]}{p[h^t|\hat{\theta}]} > \alpha\geq 1\]

Where \(h^t\) is the history of outcomes up to time \(t\) and \(\alpha\)
is the switching threshold. By keeping track of the likelihood ratio,
the agent can perform a \emph{Bayesian hypothesis test} and adopt the
Dogmatic belief that best fits the
data.\footnote{In a related problem @Schwarstein2021 proposes a similar updating procedure wich relies on the Bayesian 
hypothesis test.} Notice that if \(\alpha \to \infty\), the behavior of
the switcher will be indistinguishable from that of the Dogmatic
modeler. In this sense, the switcher is a generalization of the dogmatic
type.

By allowing the agent to keep track of the likelihoods and switching to
an alternative type, the switcher can avoid the self-confirming
equilibrium. However, if the prior belief on \(omega\) is sufficiently
tight around a self-defeating equilibrium, the switcher might look
identical to the dogmatic even in a case where \(\alpha\) is not too
large. This happens because under the agent's prior, the likelihood
ratio is unlikely to grow as fast as it is needed to escape the
self-defeating equilibrium. In this cases we say that the misspecified
belief is persistent. Hence, in order to be able to determine which of
these two models provides a better explanation of the observed
behaviors, we must focus on cases in which the prior is diffused enough
and the sequence of realized outcomes is such that the self-defeating
equilibrium would not persist.

\hypertarget{framework-2}{%
\section{Framework 2}\label{framework-2}}

The agent is still of some type \(\theta^* \in \Theta\) and the state is
\(\omega \sim F\). In this case the agent chooses an action \(a\in A\)
and observes a binary outcome that is either a success or a failure.
Denote the outcome by \(o \in {s,f}\). The probability of observing a
success is ingreasing in \(\theta^*\) and in \(\omega\). Whenever the
agent observes a succes, he gets a payoff \(v>0\) and whenever the
outcome is a failure, normalize the payoff to 0. In addition, the
probability of success is such that for each state, there is a unique
optimal action that maximizes the agent's expected payoff.

Two nested theories that have been widely studied fall within this
framework: Fully Bayesian updating and self-serving attribution bias. I
explain each of these classical models of belief updating in what
follows

\hypertarget{the-bayesian}{%
\subsection{The Bayesian}\label{the-bayesian}}

A Bayesian agent simultaneously updates their beliefs about \(\theta\)
and \(\omega\) by using Bayes' rule. The posterior odds at period t
about \(\theta\) after observing an outcome are given by: \[
\frac{p_{t}[\theta_H|\text{outcome}]}{p_{t}[\theta_M|\text{outcome}]} = 
      \frac{p[\text{outcome}|\theta_H]p_{t-1}[\theta_H]}{p[\text{outcome}|\theta_M]p_{t-1}[\theta_M]}
\] and \[
\frac{p_{t}[\theta_M|\text{outcome}]}{p_{t}[\theta_L|\text{outcome}]} = 
      \frac{p[\text{outcome}|\theta_M]p_{t-1}[\theta_M]}{p[\text{outcome}|\theta_L]p_{t-1}[\theta_L]}
\]

Where \(p_{t-1}\) is the prior at period \(t\) and
\(p[\text{outcome}|\theta] = \sum_{\omega} p[\text{outcome}|\theta, \omega, e]p_{t-1}(\omega)\)
is the probability of observing the outcome given the agent's type and
the effort chosen. The update is symmetric for \(\omega\).

Bayesian agents always choose the effort level that maximizes their flow
payoff by taking expectations over their prior beliefs about \(\theta\)
and \(\omega\). Since agents are myopic, even though all the parameters
could be identified with enough variation in choices, although the
Bayesian agent is the closest to a fully rational agent discussed here,
they might not learn their true type. This happens because, by being
myopic, they do not internalize the tradeoff between flow payoff and
learning. This can result in too little experimentation to learn their
true type. An alternative to this approach is given by
\citet{Hestermann2021} and is discussed with the results.

\hypertarget{the-self-serving-bayesian}{%
\subsection{The Self-Serving Bayesian}\label{the-self-serving-bayesian}}

A self-serving bayesian is an agent who uses a biased version of Bayes
rule to update his beliefs.He will update his beliefs about the state
\(\omega\) and his type \(\theta\) by over-attributing successes to a
high value of \(\theta\) and under-estimating the role of higher
\(\omega\). Similarly, he will attribute failure to a low state to a
greter degree than an unbiased agent would. To model the self-serving
attribution bias, I take the approach of \citet{benjamin2019}, where the
posterior is given by:

\[
p_{t}[\theta_H|\text{outcome}] = 
\frac{p[\text{outcome}|\theta]^{c_s^{\theta}\mathbb{I}\{\text{success}\}+c_f^{\theta}\mathbb{I}\{\text{failure}\}}p_{t-1}[\theta]}
{\sum_{\theta'\in\Theta}p[\text{outcome}|\theta']^{c_s^{\theta}\mathbb{I}\{\text{success}\}+c_f^{\theta}\mathbb{I}\{\text{failure}\}}p_{t-1}[\theta']}
\]

\(c_s^{\theta}\) and \(c_f^{\theta}\) are the self-serving attribution
bias parameters for the agent's type \(\theta\). If
\(c_s^{\theta} = c_f^{\theta} = 1\), the agent is unbiased and the
update is the same as the Bayesian update. On the other hand, if
\(c_s^{\theta} > c_f{\theta}\) the agent over-attributes success to
their type and under-attributes failure to their type\footnote{
  notice that the values of $c_s^{\theta}$ and $c_f^{\theta}$ are not restricted to be greater than 1. If they are both equal to 
  each other but less (more) than one, then the bias is simply underinference (overinference).}.

The update for \(\omega\) is analogous but with \(c_f^{\omega}\) and
\(c_s^{\omega}\) instead of \(c_f^{\theta}\) and \(c_s^{\theta}\) and
the bias is present whenever \(c_f^{\omega} > c_s^{\omega}\). That is,
the agent over-attributes failure to a low state relative the higher
states and under-attributes success to a low state relative to the
higher states.

\hypertarget{a-unifying-example}{%
\section{A Unifying Example}\label{a-unifying-example}}

The agent can be of one of 3 types:
\(\theta \in \{\theta_L, \theta_M, \theta_H\}\) with
\(\theta_H > \theta_M > \theta_L\). They face an unknown exogenous
success rate \(\omega \in \{\omega_L, \omega_M, \omega_H\}\) with
\(\omega_H>\omega_M>\omega_L\). Each of the values of \(\omega\) is
realized with equal probability. The agent knows the distribution of
\(\omega\) but not its realized value.

Denote the true type by \(\theta^*\) and the true state by \(\omega^*\).
The agent holds some prior belief about \(\theta\)
\footnote{which is potentially misspecified as in the dogmatic and switcher cases discussed above}
and chooses a binary gamble \(e in \{e_L, e_M, e_H\}\). The agent
observes whether the gamble is a success or a failure and gets a payoff
of \(1\) and if it is a success; they get \(0\) otherwise.

The probability of success is increasing in both \(\theta\) and
\(\omega\) and is fully described by the following table:

\begin{tabular}{ c|c|c|c|}
  
  \multicolumn{1}{c}{} & \multicolumn{1}{c}{$\omega_H$} & \multicolumn{1}{c}{$\omega_M$} & \multicolumn{1}{c}{$\omega_L$}\\
  \cline{2-4}
  $e_H$ & 50 & 20 & 2 \\
  \cline{2-4}
  $e_M$ & 45 & 30 & 7 \\
  \cline{2-4}
  $e_L$ & 40 & 25 & 20 \\

  \cline{2-4}
  \multicolumn{1}{c}{} & \multicolumn{1}{c}{} & \multicolumn{1}{c}{$\theta_L$} & \multicolumn{1}{c}{}\\
\end{tabular}
\hspace{.3cm} 
\begin{tabular}{ c|c|c|c|}
  
  \multicolumn{1}{c}{} & \multicolumn{1}{c}{$\omega_H$} & \multicolumn{1}{c}{$\omega_M$} & \multicolumn{1}{c}{$\omega_L$}\\
  \cline{2-4}
  $e_H$ & 80 & 50 & 5 \\
  \cline{2-4}
  $e_M$ & 69 & 65 & 30 \\
  \cline{2-4}
  $e_L$ & 65 & 45 & 40 \\
  \cline{2-4}
  \multicolumn{1}{c}{} & \multicolumn{1}{c}{} & \multicolumn{1}{c}{$\theta_M$} & \multicolumn{1}{c}{}\\
\end{tabular}
\hspace{.3cm} 
\begin{tabular}{ c|c|c|c|}
  
  \multicolumn{1}{c}{} & \multicolumn{1}{c}{$\omega_H$} & \multicolumn{1}{c}{$\omega_M$} & \multicolumn{1}{c}{$\omega_L$}\\
  \cline{2-4}
  $e_H$ & 98 & 65 & 25 \\
  \cline{2-4}
  $e_M$ & 80 & 69 & 35 \\
  \cline{2-4}
  $e_L$ & 75 & 55 & 45 \\
  \cline{2-4}
  \multicolumn{1}{c}{} & \multicolumn{1}{c}{} & \multicolumn{1}{c}{$\theta_H$} & \multicolumn{1}{c}{}\\
\end{tabular}

Conditional on a type, the agent's flow payoff is maximized by choosing
the gamble that matches the state. For example, if the value of
\(\omega\) is \(\omega_H\), the agent's flow payoff is maximized by
choosing \(e_H\) and if the state is \(\omega_L\) the flow payoff is
maximized by choosing gamble \(e_L\), regardless of the value of
\(\theta\). The agent myopically chooses gambles every period to
maximize the flow payoff for \(T<\infty\) periods.

After observing the outcome of each gamble, the agent updates their
beliefs using some procedure and moves on to the next period.

Notice that both \(\theta\) and \(\omega\) can be identified from the
outcomes if enough variation in the effort choices exists. This can be
seen by confirming that there is no pair of \(\theta\) and \(\omega\)
such that the probability of success is the same for all effort choices.
Thus, by changing the effort choice, the agent can learn both their type
and the state if they observe enough outcomes.

In this example, for an agent with a dogmatic belief about their type, a
self-defeating equilibrium is one in which the agent chooses an effort
level that, under the true \(\theta\), yields a frequency of success
that is consistent with the agent's misspecified belief. That is
\(P[\text{sucess}|\theta^*, e^*] = P[\text{sucess}|\hat{\theta}, e^*]\)
where \(e^*\) is the agent's myopic optimal choice.

In the data-generating process described above, there are five such
equilibria. For example, if the agent is of type \(\theta_M\) but
mistakenly believes that he is of type \(\hat{\theta}=\theta_H\) and the
and \(\omega^* = \omega_M\), when the effort chosen is \(e_L\), the
agent will observe a success with 45\% chance. Because the agent
dogmatically believes that their type is high, they will erroneously
conclude that the rate is \(\omega_L\). Under this belief, the optimal
action is \(e_L\) which will continue to generate successes with \(45%
\) probability, further reinforcing the incorrect belief. By doing so,
the agent forgoes the payoff from gamble \(e_M\) which would yield a
success with \(65\%\) chance.

By including self-confirming equilibria, the example captures the forces
from each of the updating mechanisms discussed in the previous section
and allows for direct comparison of all the theories. For realizations
of \((\theta, \omega)\) for which there are self-confirming equilibria,
the dogmatic agent will fall into the trap whereas the switcher will be
able to escape it. Similarly, an agent with self-attribution bias will
update their beliefs differently from an unbiased Bayesian, leading them
to choose different gambles. I exploit such cases in order to test which
model is a better fit for how subjects behave in a laboratory
experiment.
\footnote{because the setting does not match that of @Heidhues2018, there will be situationf for which the theory 
does not provie a prediction. If such cases arise in the lab, they will not be used for the analysis. However, whether a misspecified 
belief persists or not for the switcher, depends highly on the realized history of signals that he gets.}
In what follows I explain the details of how this example was
implemented in the lab.

\hypertarget{experimental-design}{%
\section{Experimental Design}\label{experimental-design}}

I recruited XXX undergraduate subjects from the CESS lab at NYU who
participated in an in-person experiment. Sessions lasted approximately
XXX hours and subjects earned an average payment of XXX. The experiment
was programmed using oTree \citep{chen2016otree}.

The experiment consisted of 2 treatments: the \emph{ego-relevant}
condition and the \emph{stereotype} condition. Subjects participated in
only one of the treatments. Treatments were randomly assigned at the
session level. The tasks were identical across treatments, except for
parameter \(\theta\). In the ego-relevant condition \(\theta\) is the
subject's own performance in a quiz, while in the stereotype condition,
it is the performance of a randomly selected subject from another
session.

The experiment had 3 parts. In Part 1 subjects had 2 minutes to answer
as many multiple-choice questions as they could from a 20-question quiz.
They did this for quizzes on 6 different topics. The topics were: Math,
Verbal Reasoning, Pop-culture and Art, Science and Technology, US
Geography, and Sports and Video Games. In this part, they did not know
how many questions were available and they were given no feedback.

After taking all 6 quizzes, they proceeded to part 2 where they were
asked to guess their score on each of them. In the stereotype treatment
they were additionally asked to guess the score of a randomly drawn
participant from a previous session. All they knew about the other
participant was their gender identity and whether they were US nationals
or not. For each guess they had three score options: Low-Score (5 or
fewer correct answers), Mid-Score (between 6 and 15 correct answers),
High-Score (16 or more). Each of the score categories correspond to
\(\theta_L\), \(\theta_M\), and \(\theta_H\) respectively. They were
also asked to say how confident they felt about their choices. They had
4 possible answers: ``it was a random guess'', ``there is another
equally likely score'', ``I am pretty sure'', ``I am completely sure''.
These 4 answers are mapped to priors that place probabilities .33, .50,
.75, and 1 to the chosen type. The remaining probability is split
equally among the other two types. Questions in part 2 were not
incentivized, but subjects were told that providing an accurate answer
would increase their chances of earning more money in the last part of
the experiment.

The purpose of part 2 is to classify subjects into overconfident,
underconfident and correctly specified. If a subject guesses their score
to be in a higher (lower) category than their true score, they are
overconfident (underconfident); if they guess their score to be in the
same category as their true score, they are correctly specified. This
classification is done for each of the 6 topics separately.

Finally, in part 3 subjects completed a belief updating task for each of
the quizzes. Before starting the task they were reminded of their guess
for the score. In the ego treatment they were reminded of their guess
about themselves and in the stereotype treatment they were reminded of
their guess about the other participant. In the stereotype treatment,
they were also reminded of the characteristics of the other participant.

For one topic at a time and in random order, they were presented with
the three gambles from the example above, and were asked to choose one
of them. The probability of success was determined by their own score in
the ego-relevant condition, and by the score of the other participant in
the stereotype condition. Subjects had access to the three probability
tables in the printout of the instructions at all times and the meaning
of each cell was explained in detail.

In the interphase, they had to choose which of the 3 tables they wanted
to see before entering their choice in it. This was done as an
alternative to a belief elicitation in each round. I take their choice
of table to be a signal for their beliefs about the underlying type. I
chose not to elicit the beliefs at each round to stay true to the forces
in framework 1.

Once they have entered their choice, they observe a sample of 10
outcomes from the gamble they chose. After observing the outcomes, they
returned to the choice screen and entered a new choice. In the choice
screen subjects had access to the entire history of gambles and outcomes
for that task as well as a summary of the outcomes so far. Once they
entered 11 gambles (and observed 110 outcomes), they moved on to the
next topic and repeated the same procedure. They all did this for all 6
topics.

At the end of the experiment, one of the 6 topics was randomly selected
to determine the payment. They earned \(\$0.20\) for each correct answer
in the quiz, and for each success in part 3.

Randomness is controlled throughout the experiment and sessions by
setting a seed at the beggining of the first session. The seed was drawn
at random and remained fixed for all sessions. By doing this I ensure
that any two subjects who have the same type and face the same exogenous
rate will observe the same outcomes and thus, if they use the same
updating procedure, they should be choosing the same gambles. This
design feature allows me to identify differences in updating procedures
across subjects.

\hypertarget{analysis}{%
\section{Analysis}\label{analysis}}

\renewcommand\refname{Conclusion}
  \bibliography{references.bib}

\end{document}
