\documentclass[aspectratio=169]{beamer}
\usetheme{metropolis}
\usepackage{geometry}
\usepackage{amsmath}
\usepackage{graphicx}
\graphicspath{ {./figures2/} }
\usepackage{amsfonts}
\usepackage{amssymb}
\usepackage{setspace}
\usepackage{theorem}
\usepackage{natbib}
\usepackage{mathtools}
\usepackage{cite}
\usepackage{natbib}
\usepackage{setspace}
\usepackage[utf8]{inputenc}
\usepackage[english]{babel}
\usepackage{array}
\usepackage{caption}
\usepackage{siunitx}
\usepackage[normalem]{ulem}
\usepackage{multirow}
\usepackage{hhline}
\usepackage{calc}
\usepackage{tabularx}
\usepackage{threeparttable}
\usepackage{wrapfig}
\usepackage{adjustbox}
\usepackage{hyperref}
\usepackage{tikz}
\usepackage[table]{xcolor}
\usepackage{colortbl}
\usepackage{appendixnumberbeamer}

\title{Learning with Misspecified Models:\\  
The case of overconfidence}
\author{Jimena Galindo}

  
\begin{document}

\frame{\titlepage}

\begin{frame}{Overconfidence}
    \textbf{Overestimation}: Belief that type is higher than it truly is\\
    \begin{itemize}
        \item \textit{e.g. Believing you have an IQ of 150 when it is actually 100}
    \end{itemize}
    \bigskip
    \pause
    Seems to be persistent in various settings. 
    \begin{itemize} 
        \item Excess entry of entrepreneurs (Camerer and Lovallo, 1999)
        \item Suboptimal genetic testing and savings (Oster et al. 2013)
        \item  Workers overestimate their productivity (Hoffman and Burks, 2020)
    \end{itemize}
    \bigskip
    \alert{Ultimately it leads to costly choices}
    
\end{frame}

\begin{frame}{Models of Learning}
    Focus on setting with 2 parameters:
    \begin{itemize}
        \item An \alert{\textbf{Ego-Relevant}} parameter
        \item An \alert{\textbf{Exogenous}} parameter
    \end{itemize}
    \textit{For instance skill and luck when playing a game}\\
    \bigskip
    \pause
    Some of the assumptions that theory has incorporated to rationalize overconfidence are:
    \begin{itemize}
        \item Dogmatism
        \item Paradigm shifts
        \item Motivated beliefs
        \item Myopic Bayesian
    \end{itemize}
\end{frame}

\begin{frame}{Four Theories of Misspecified Learning}
    
    \begin{enumerate}

        \item \textbf{Self-defeating equilibrium} (Heidhues et al. (2018)) \\
        \begin{itemize}
            \item Bayesian about exogenous parameters
            \item Dogmatic about ego-relevant parameters
        \end{itemize}
        \bigskip

        \item \textbf{Bayesian hypothesis testing} (Schwarstein and Sunderam (2021), Ba (2022))\\
        \begin{itemize}
            \item Bayesian about exogenous parameters 
            \item Paradigm shift for ego-relevant parameters
        \end{itemize}
        \bigskip

        \item \textbf{Motivated Beliefs / Self-Attribution Bias} (Brunnermeier and Parker (2005), Bracha and Brown (2012)) \\
        \begin{itemize}
            \item Optimally biased updating
            \item Utility from held beliefs 
        \end{itemize}
        \bigskip

        \item \textbf{Myopic Bayesian} (Hestermann and Le Yaouanq, (2021))\\
        \begin{itemize}
            \item Bayesian about both 
            \item Maximizes flow utility only
        \end{itemize}

    \end{enumerate}
    
\end{frame}

\begin{frame}{Questions}
    Which of the proposed theories gives a better explanation of behavior?\\
    \bigskip
    Do the theories apply only to misspecifications about ego-relevant parameters?\\
    \begin{itemize}
        \item Can the same theories explain the prevalence of stereotypes?
        
    \end{itemize}
\end{frame}


\begin{frame}{An Example (from Heidhues et al. (2018))}
    A student has unknown \textbf{intrinsic ability} $\theta^*$ (\alert{ego-relevant parameter})\\ 
    \bigskip
    They choose a level of \textbf{effort} $e\geq 0$ (\alert{choice}) \\
    \bigskip
    Effort and ability are evaluated by a \textbf{grading system} $\omega$ (\alert{exogenous parameter})\\
    \bigskip 
    The student wants to maximize:\\
        $$u(e) = (\theta^* + e)\omega-\frac{1}{2}e^2 +\varepsilon$$\\
    
    \bigskip
   \textbf{Regardless of their own type and of their beliefs about it, they should choose $e^*(\omega)=\omega$}\\

\end{frame}

\begin{frame}{Learning is Possible}
    This exercise is repeated for $t=0, 1, ...$
        $$y_t = (\theta^* + e_t)\omega-\frac{1}{2}e_t^2 +\varepsilon_t$$
    
    Note that both parameters are identified in this setting:\\
    \bigskip
    
    \begin{itemize}
        \item Choosing $\hat{e}$ and $\hat{e}+1$ over multiple periods allows identification of $\omega$\\
        \bigskip
        \item Once $\omega$ is known, $\theta$ can be backed out\\
     \end{itemize}
    \bigskip
    Why do people not learn the true values of the parameters?
\end{frame}

\begin{frame}{Preview of Results}

    From the proposed mechanisms:
    \begin{itemize}
        \item \alert{Dogmatism} and \alert{Bayesian Updating} do not seem to explain the behavior
        \bigskip
        \item Some evidence supporting \alert{Hypothesis Testing}
        \bigskip
        \item Most evidence supporting \alert{Motivated Beliefs}
        \bigskip
        \item Biased updating about others (but for potentially different reasons)
    \end{itemize}

\end{frame}

\begin{frame}{Roadmap}
    \begin{enumerate}
        \item Unifying Framework\\
        \bigskip
        \item Mechanisms and Predictions\\
        \bigskip
        \item Experimental Design\\
        \bigskip
        \item The Data\\
        \bigskip
        \item Results\\
    \end{enumerate}
\end{frame}

\section*{Framework}

\begin{frame}{A Unifying Framework}

\textbf{Ego-relevant paremeter}: $\theta \in \{\theta_H, \theta_M, \theta_L\}$\\
\bigskip

\textbf{Exogenous parameter}: $\omega \in \{\omega_H, \omega_M, \omega_L\}$
with $p(\omega_k)=1/3$ \\
\bigskip

\textbf{Choices}: $e \in \{e_H, e_M, e_L\}$\\
\bigskip

\textbf{Binary Outcomes}: $s_t \in\{\text{success, failure}\}$ with  $p\left[success|e, \omega, \theta\right]$ and p is an order-preserving transformation of $u(x)$


\end{frame}


\begin{frame}{The Data Generating Process}
    The probability of success is given by:\\
    \bigskip
    \centering
    \begin{tabular}{ c|c|c|c|}
    
    \multicolumn{1}{c}{} & \multicolumn{1}{c}{$\omega_H$} & \multicolumn{1}{c}{$\omega_M$} & \multicolumn{1}{c}{$\omega_L$}\\
    \cline{2-4}
    $e_H$ & 50 & 20 & 2 \\
    \cline{2-4}
    $e_M$ & 45 & 30 & 7 \\
    \cline{2-4}
    $e_L$ & 40 & 25 & 20 \\
    \cline{2-4}
    \multicolumn{1}{c}{} & \multicolumn{1}{c}{} & \multicolumn{1}{c}{$\theta_L$} & \multicolumn{1}{c}{}\\
    \end{tabular}
    \hspace{.3cm} % adjust this value to set the space between tables
    \begin{tabular}{ c|c|c|c|}
    
    \multicolumn{1}{c}{} & \multicolumn{1}{c}{$\omega_H$} & \multicolumn{1}{c}{$\omega_M$} & \multicolumn{1}{c}{$\omega_L$}\\
    \cline{2-4}
    $e_H$ & 80 & 50 & 5 \\
    \cline{2-4}
    $e_M$ & 69 & 65 & 30 \\
    \cline{2-4}
    $e_L$ & 65 & 45 & 40 \\
    \cline{2-4}
    \multicolumn{1}{c}{} & \multicolumn{1}{c}{} & \multicolumn{1}{c}{$\theta_M$} & \multicolumn{1}{c}{}\\
    \end{tabular}
    \hspace{.3cm} % adjust this value to set the space between tables
    \begin{tabular}{ c|c|c|c|}
    
    \multicolumn{1}{c}{} & \multicolumn{1}{c}{$\omega_H$} & \multicolumn{1}{c}{$\omega_M$} & \multicolumn{1}{c}{$\omega_L$}\\
    \cline{2-4}
    $e_H$ & 98 & 65 & 25 \\
    \cline{2-4}
    $e_M$ & 80 & 69 & 35 \\
    \cline{2-4}
    $e_L$ & 75 & 55 & 45 \\
    \cline{2-4}
    \multicolumn{1}{c}{} & \multicolumn{1}{c}{} & \multicolumn{1}{c}{$\theta_H$} & \multicolumn{1}{c}{}\\
    \end{tabular}

\end{frame}

\begin{frame}{The Data Generating Process}
    \centering
\begin{tabular}{ c|c|c|c|}
  
  \multicolumn{1}{c}{} & \multicolumn{1}{c}{$\omega_H$} & \multicolumn{1}{c}{$\omega_M$} & \multicolumn{1}{c}{$\omega_L$}\\
  \cline{2-4}
  $e_H$ & \cellcolor{blue!25}50 & 20 & 2 \\
  \cline{2-4}
  $e_M$ & 45 & \cellcolor{blue!25}30 & 7 \\
  \cline{2-4}
  $e_L$ & 40 & 25 & \cellcolor{blue!25}20 \\
  \cline{2-4}
  \multicolumn{1}{c}{} & \multicolumn{1}{c}{} & \multicolumn{1}{c}{$\theta_L$} & \multicolumn{1}{c}{}\\
\end{tabular}
\hspace{.3cm} % adjust this value to set the space between tables
\begin{tabular}{ c|c|c|c|}
  
  \multicolumn{1}{c}{} & \multicolumn{1}{c}{$\omega_H$} & \multicolumn{1}{c}{$\omega_M$} & \multicolumn{1}{c}{$\omega_L$}\\
  \cline{2-4}
  $e_H$ & \cellcolor{blue!25}80 & 50 & 5 \\
  \cline{2-4}
  $e_M$ & 69 & \cellcolor{blue!25}65 & 30 \\
  \cline{2-4}
  $e_L$ & 65 & 45 & \cellcolor{blue!25}40 \\
  \cline{2-4}
  \multicolumn{1}{c}{} & \multicolumn{1}{c}{} & \multicolumn{1}{c}{$\theta_M$} & \multicolumn{1}{c}{}\\
\end{tabular}
\hspace{.3cm} % adjust this value to set the space between tables
\begin{tabular}{ c|c|c|c|}
  
  \multicolumn{1}{c}{} & \multicolumn{1}{c}{$\omega_H$} & \multicolumn{1}{c}{$\omega_M$} & \multicolumn{1}{c}{$\omega_L$}\\
  \cline{2-4}
  $e_H$ & \cellcolor{blue!25}98 & 65 & 25 \\
  \cline{2-4}
  $e_M$ & 80 & \cellcolor{blue!25}69 & 35 \\
  \cline{2-4}
  $e_L$ & 75 & 55 & \cellcolor{blue!25}45 \\
  \cline{2-4}
  \multicolumn{1}{c}{} & \multicolumn{1}{c}{} & \multicolumn{1}{c}{$\theta_H$} & \multicolumn{1}{c}{}\\
\end{tabular}

\end{frame}


\begin{frame}{The Data Generating Process}
\begin{center}
\begin{tikzpicture}
  \draw[<-] (2,0) -- (4,0);
  \draw[<-] (6,0) -- (8,0);
  \draw[<-] (10,0) -- (12,0);
\end{tikzpicture}
\end{center}

\bigskip
\centering
\begin{tabular}{ c|c|c|c|}
  
  \multicolumn{1}{c}{} & \multicolumn{1}{c}{$\omega_H$} & \multicolumn{1}{c}{$\omega_M$} & \multicolumn{1}{c}{$\omega_L$}\\
  \cline{2-4}
  $e_H$ & \cellcolor{blue!25}50 & 20 & 2 \\
  \cline{2-4}
  $e_M$ & 45 & \cellcolor{blue!25}30 & 7 \\
  \cline{2-4}
  $e_L$ & 40 & 25 & \cellcolor{blue!25}20 \\
  \cline{2-4}
  \multicolumn{1}{c}{} & \multicolumn{1}{c}{} & \multicolumn{1}{c}{$\theta_L$} & \multicolumn{1}{c}{}\\
\end{tabular}
\hspace{.3cm} % adjust this value to set the space between tables
\begin{tabular}{ c|c|c|c|}
  
  \multicolumn{1}{c}{} & \multicolumn{1}{c}{$\omega_H$} & \multicolumn{1}{c}{$\omega_M$} & \multicolumn{1}{c}{$\omega_L$}\\
  \cline{2-4}
  $e_H$ & \cellcolor{blue!25}80 & 50 & 5 \\
  \cline{2-4}
  $e_M$ & 69 & \cellcolor{blue!25}65 & 30 \\
  \cline{2-4}
  $e_L$ & 65 & 45 & \cellcolor{blue!25}40 \\
  \cline{2-4}
  \multicolumn{1}{c}{} & \multicolumn{1}{c}{} & \multicolumn{1}{c}{$\theta_M$} & \multicolumn{1}{c}{}\\
\end{tabular}
\hspace{.3cm} % adjust this value to set the space between tables
\begin{tabular}{ c|c|c|c|}
  
  \multicolumn{1}{c}{} & \multicolumn{1}{c}{$\omega_H$} & \multicolumn{1}{c}{$\omega_M$} & \multicolumn{1}{c}{$\omega_L$}\\
  \cline{2-4}
  $e_H$ & \cellcolor{blue!25}98 & 65 & 25 \\
  \cline{2-4}
  $e_M$ & 80 & \cellcolor{blue!25}69 & 35 \\
  \cline{2-4}
  $e_L$ & 75 & 55 & \cellcolor{blue!25}45 \\
  \cline{2-4}
  \multicolumn{1}{c}{} & \multicolumn{1}{c}{} & \multicolumn{1}{c}{$\theta_H$} & \multicolumn{1}{c}{}\\
\end{tabular}

\begin{center}
    \resizebox{0.6\linewidth}{!}{\vector(1,0){300}}
  \end{center}
    
\end{frame}


\begin{frame}{A Stable Misspecified Belief}

\centering
\begin{tabular}{ c|c|c|c|}
  
  \multicolumn{1}{c}{} & \multicolumn{1}{c}{$\omega_H$} & \multicolumn{1}{c}{$\omega_M$} & \multicolumn{1}{c}{$\omega_L$}\\
  \cline{2-4}
  $e_H$ & \cellcolor[HTML]{b84f79}50 & 20 & 2 \\
  \cline{2-4}
  $e_M$ & 45 & 30 & 7 \\
  \cline{2-4}
  $e_L$ & 40 & 25 & 20 \\
  \cline{2-4}
  \multicolumn{1}{c}{} & \multicolumn{1}{c}{} & \multicolumn{1}{c}{$\theta_L$} & \multicolumn{1}{c}{}\\
\end{tabular}
\hspace{.3cm} % adjust this value to set the space between tables
\begin{tabular}{ c|c|c|c|}
  
  \multicolumn{1}{c}{} & \multicolumn{1}{c}{$\omega_H$} & \multicolumn{1}{c}{$\omega_M$} & \multicolumn{1}{c}{$\omega_L$}\\
  \cline{2-4}
  $e_H$ & 80 & \cellcolor[HTML]{b84f79}50 & 5 \\
  \cline{2-4}
  $e_M$ & 69 &\cellcolor[HTML]{f09ebe}65 & 30 \\
  \cline{2-4}
  $e_L$ & 65 & \cellcolor[HTML]{f09ebe}45 & 40 \\
  \cline{2-4}
  \multicolumn{1}{c}{} & \multicolumn{1}{c}{} & \multicolumn{1}{c}{$\theta_M$} & \multicolumn{1}{c}{}\\
\end{tabular}
\hspace{.3cm} % adjust this value to set the space between tables
\begin{tabular}{ c|c|c|c|}
  
  \multicolumn{1}{c}{} & \multicolumn{1}{c}{$\omega_H$} & \multicolumn{1}{c}{$\omega_M$} & \multicolumn{1}{c}{$\omega_L$}\\
  \cline{2-4}
  $e_H$ & 98 & 65 & 25 \\
  \cline{2-4}
  $e_M$ & 80 & 69 & 35 \\
  \cline{2-4}
  $e_L$ & 75 & 55 & 45 \\
  \cline{2-4}
  \multicolumn{1}{c}{} & \multicolumn{1}{c}{} & \multicolumn{1}{c}{$\theta_H$} & \multicolumn{1}{c}{}\\
\end{tabular}
\end{frame}


\begin{frame}{The Stable Beliefs}

    \centering
    \begin{tabular}{ c|c|c|c|}
    
    \multicolumn{1}{c}{} & \multicolumn{1}{c}{$\omega_H$} & \multicolumn{1}{c}{$\omega_M$} & \multicolumn{1}{c}{$\omega_L$}\\
    \cline{2-4}
    $e_H$ & \cellcolor[HTML]{b84f79}50 & 20 & 2 \\
    \cline{2-4}
    $e_M$ & 45 & \cellcolor[HTML]{5f94b8}30 & 7 \\
    \cline{2-4}
    $e_L$ & \cellcolor[HTML]{69a35b}40 & 25 & 20 \\
    \cline{2-4}
    \multicolumn{1}{c}{} & \multicolumn{1}{c}{} & \multicolumn{1}{c}{$\theta_L$} & \multicolumn{1}{c}{}\\
    \end{tabular}
    \hspace{.3cm} % adjust this value to set the space between tables
    \begin{tabular}{ c|c|c|c|}
    
    \multicolumn{1}{c}{} & \multicolumn{1}{c}{$\omega_H$} & \multicolumn{1}{c}{$\omega_M$} & \multicolumn{1}{c}{$\omega_L$}\\
    \cline{2-4}
    $e_H$ & 80 & \cellcolor[HTML]{b84f79}50 & 5 \\
    \cline{2-4}
    $e_M$ & \cellcolor[HTML]{fab143}69 & 65 & \cellcolor[HTML]{5f94b8}30 \\
    \cline{2-4}
    $e_L$ & 65 & \cellcolor[HTML]{9662f0}45 & \cellcolor[HTML]{69a35b}40 \\
    \cline{2-4}
    \multicolumn{1}{c}{} & \multicolumn{1}{c}{} & \multicolumn{1}{c}{$\theta_M$} & \multicolumn{1}{c}{}\\
    \end{tabular}
    \hspace{.3cm} % adjust this value to set the space between tables
    \begin{tabular}{ c|c|c|c|}
    
    \multicolumn{1}{c}{} & \multicolumn{1}{c}{$\omega_H$} & \multicolumn{1}{c}{$\omega_M$} & \multicolumn{1}{c}{$\omega_L$}\\
    \cline{2-4}
    $e_H$ & 98 & 65 & 25 \\
    \cline{2-4}
    $e_M$ & 80 & \cellcolor[HTML]{fab143}69 & 35 \\
    \cline{2-4}
    $e_L$ & 75 & 55 & \cellcolor[HTML]{9662f0}45 \\
    \cline{2-4}
    \multicolumn{1}{c}{} & \multicolumn{1}{c}{} & \multicolumn{1}{c}{$\theta_H$} & \multicolumn{1}{c}{}\\
    \end{tabular}
\end{frame}

\section*{Mechanisms and Predictions}

\begin{frame}{An Example}
    \begin{itemize}
        \item True type is $\theta_M$ \\
        \bigskip
        \item True parameter is $\omega_M$ $\rightarrow$ the student believes it is uniformly distributed\\
        \end{itemize}

        \centering
    \begin{tabular}{ c|c|c|c|}
    
    \multicolumn{1}{c}{} & \multicolumn{1}{c}{$\omega_H$} & \multicolumn{1}{c}{$\omega_M$} & \multicolumn{1}{c}{$\omega_L$}\\
    \cline{2-4}
    $e_H$ & 50 & 20 & 2 \\
    \cline{2-4}
    $e_M$ & 45 & 30 & 7 \\
    \cline{2-4}
    $e_L$ & 40 & 25 & 20 \\
    \cline{2-4}
    \multicolumn{1}{c}{} & \multicolumn{1}{c}{} & \multicolumn{1}{c}{$\theta_L$} & \multicolumn{1}{c}{}\\
    \end{tabular}
    \hspace{.3cm} % adjust this value to set the space between tables
    \begin{tabular}{ c|c|c|c|}
    
    \multicolumn{1}{c}{} & \multicolumn{1}{c}{$\omega_H$} & \multicolumn{1}{c}{$\omega_M$} & \multicolumn{1}{c}{$\omega_L$}\\
    \cline{2-4}
    $e_H$ & 80 & \cellcolor{blue!25}50 & 5 \\
    \cline{2-4}
    $e_M$ & 69 & \cellcolor{blue!25}65 & 30 \\
    \cline{2-4}
    $e_L$ & 65 & \cellcolor{blue!25}45 & 40 \\
    \cline{2-4}
    \multicolumn{1}{c}{} & \multicolumn{1}{c}{} & \multicolumn{1}{c}{$\theta_M$} & \multicolumn{1}{c}{}\\
    \end{tabular}
    \hspace{.3cm} % adjust this value to set the space between tables
    \begin{tabular}{ c|c|c|c|}
    
    \multicolumn{1}{c}{} & \multicolumn{1}{c}{$\omega_H$} & \multicolumn{1}{c}{$\omega_M$} & \multicolumn{1}{c}{$\omega_L$}\\
    \cline{2-4}
    $e_H$ & 98 & 65 & 25 \\
    \cline{2-4}
    $e_M$ & 80 & 69 & 35 \\
    \cline{2-4}
    $e_L$ & 75 & 55 & 45 \\
    \cline{2-4}
    \multicolumn{1}{c}{} & \multicolumn{1}{c}{} & \multicolumn{1}{c}{$\theta_H$} & \multicolumn{1}{c}{}\\
    \end{tabular}
    
\end{frame}


\begin{frame}{The Dogmatic Modeler}

    Holds a degenerate belief: type is $\hat{\theta}$ with probability 1\\
    \bigskip
    Their belief is potentially misspecified:\\
    \begin{itemize}
        \item Overconfident if $\hat{\theta}>\theta^*$
        \item Underconfident if $\hat{\theta}<\theta^*$
    \end{itemize}
    \bigskip
    Updates $p_t(\omega)$ using Bayes Rule

    $$p_{t+1}(\omega|s, \hat{\theta}) = \frac{p_t(s_t|\omega, \hat{\theta})p_{t}(\omega)}{\sum_{\omega'}p_t(s_t|\omega', \hat{\theta})p_{t}(\omega')}$$
    
\end{frame}

\begin{frame}{The Dogmatic Modeler: Mechanism}
    A student who dogmatically believes he is $\theta_H$ \\
    \begin{enumerate}
        \item Chooses $e_H$ and is disappointed $\rightarrow$ adjust belief about $\omega$ downward\\
        \bigskip
        \item Eventually chooses $e_M$ and is disappointed as well $\rightarrow$ adjust belief about $\omega$\\
        \bigskip
        \item Eventually chooses $e_L$ and falls into a self-confirming equilibrium
    \end{enumerate}

    \begin{center}
    \begin{tabular}{ c|c|c|c|}
    
    \multicolumn{1}{c}{} & \multicolumn{1}{c}{$\omega_H$} & \multicolumn{1}{c}{$\omega_M$} & \multicolumn{1}{c}{$\omega_L$}\\
    \cline{2-4}
    $e_H$ & 50 & 20 & 2 \\
    \cline{2-4}
    $e_M$ & 45 & 30 & 7 \\
    \cline{2-4}
    $e_L$ & 40 & 25 & 20 \\
    \cline{2-4}
    \multicolumn{1}{c}{} & \multicolumn{1}{c}{} & \multicolumn{1}{c}{$\theta_L$} & \multicolumn{1}{c}{}\\
    \end{tabular}
    \hspace{.3cm} % adjust this value to set the space between tables
    \begin{tabular}{ c|c|c|c|}
    
    \multicolumn{1}{c}{} & \multicolumn{1}{c}{$\omega_H$} & \multicolumn{1}{c}{\tikz[baseline=-0.5ex]{\node[draw=red,circle,inner sep=2pt]{$\omega_M$};}} & \multicolumn{1}{c}{$\omega_L$}\\
    \cline{2-4}
    $e_H$ & 80 & \cellcolor{blue!25}50 & 5 \\
    \cline{2-4}
    $e_M$ & 69 & \cellcolor{blue!25}65 & 30 \\
    \cline{2-4}
    $e_L$ & 65 & \cellcolor[HTML]{9662f0}45 & 40 \\
    \cline{2-4}
    \multicolumn{1}{c}{} & \multicolumn{1}{c}{} & \multicolumn{1}{c}{$\theta_M$} & \multicolumn{1}{c}{}\\
    \end{tabular}
    \hspace{.3cm} % adjust this value to set the space between tables
    \begin{tabular}{ c|c|c|c|}
    
    \multicolumn{1}{c}{} & \multicolumn{1}{c}{$\omega_H$} & \multicolumn{1}{c}{$\omega_M$} & \multicolumn{1}{c}{$\omega_L$}\\
    \cline{2-4}
    $e_H$ & 98 & 65 & 25 \\
    \cline{2-4}
    $e_M$ & 80 & 69 & 35 \\
    \cline{2-4}
    $e_L$ & 75 & 55 & \cellcolor[HTML]{9662f0}45 \\
    \cline{2-4}
    \multicolumn{1}{c}{} & \multicolumn{1}{c}{} & \multicolumn{1}{c}{\tikz[baseline=-0.5ex]{\node[draw=red,dashed, circle,inner sep=2pt]{$\theta_H$};}}  & \multicolumn{1}{c}{}\\
    \end{tabular}
    \end{center}

    \label{dogmatic}
    \action{\hyperlink{dogmaticpath}{\beamerbutton{path}}}
    
\end{frame}


\begin{frame}{The Switcher (paradigm shifts)}
    Same initial belief as the Dogmatic, but is willing to consider and alternative paradigm $\theta'$\\
    \bigskip
    Keeps track of the likelihoods of the two possible paradigms:\\
    \begin{itemize}
        \item $p_t(s_t|\cdot)$ for $\hat{\theta}$ and $\theta'$
    \end{itemize}
    \bigskip
    They switch to whichever paradigm is more likely to have generated the signals
    $$ \frac{p_t(s_t|\theta')}{p_t(s_t|\hat{\theta})}>\alpha\geq1$$
    
\end{frame}


\begin{frame}{The Switcher: Mechanism}
    

    \begin{enumerate}
        \item Chooses $e_H$ and is disappointed $\rightarrow$ adjust belief about $\omega$ downward\\
        \bigskip
        \item Eventually chooses $e_M$ and is disappointed as well $\rightarrow$ adjust belief about $\omega$\\
        \bigskip
        \item Avoids the self-defeating equilibrium if the likelihood of $\theta_M$ becomes larger than that of $\theta_H$
    \end{enumerate}

    A change in paradigm will often be accompanied with a change in effort in the opposite direction
    of the signal

    \label{switcher}
    \action{\hyperlink{switcherpath}{\beamerbutton{path}}}
    
    
\end{frame}



\begin{frame}{Self-Attribution Bias / Optimal Expectations}

    Start with a diffused prior over $(\theta, \omega)$ but updates with a bias

    $$ p_{t+1}(\theta, \omega| s_t)=\frac{p_t(s_t|\theta, \omega)^{c(\theta, \omega, s_t)}p_t(\theta, \omega)}{\sum_{(\theta', \omega')}p_t(s_t|\theta', \omega')^{c(\theta', \omega', s_t)}p_t(\theta', \omega')} $$

    Bias is such that 
    $$c(\theta_H, \omega, \text{good news}) \leq c(\theta_M, \omega, \text{good news}) \leq c(\theta_L, \omega, \text{good news})\leq1 \quad \forall \omega$$
    And
    $$c(\theta, \omega_L, \text{bad news}) \leq c(\theta, \omega_M, \text{bad news}) \leq c(\theta, \omega_H, \text{bad news})\leq1 \quad \forall \theta$$
    

\end{frame}

\begin{frame}{Self-Attribution: Mechanism}
    \begin{enumerate}
        \item Chooses $e$ that maximizes utility according to priors
        \bigskip
        \begin{itemize}
            \item Belief on $\mathbb{E}[\omega]$ deteriorates a lot after bad news $\to$ overreaction in effort
            \item Belief on $\mathbb{E}[\theta]$ increases a lot after good news $\to$ underreaction in effort (or in opposite direction) \\
        \end{itemize}
    \end{enumerate}

    \label{sspath}
    \action{\hyperlink{ss}{\beamerbutton{path}}}
    
\end{frame}


\begin{frame}{Myopic Bayesian}

    Start with a diffused prior over $(\theta, \omega)$ and updates correctly
    
    $$ p_{t+1}(\theta, \omega| s_t)=\frac{p_t(s_t|\theta, \omega)p_t(\theta, \omega)}{\sum_{(\theta', \omega')}p_t(s_t|\theta', \omega')p_t(\theta', \omega')} $$
    
    But if they start with a prior that is ``tight" 
    around a self-defeating equilibrium they will never learn 
        
\end{frame}


\begin{frame}{All Models}
    \label{Figure2}
        \begin{figure}
        \centering
        \includegraphics[scale=0.5]{models.png}
    \end{figure}     

\end{frame}

\begin{frame}{Predictions}
    \begin{figure}
        \centering
        \includegraphics[scale=0.5]{predictions.png}
    \end{figure}
\end{frame}


\section*{Experimental Design}

\begin{frame}{The Experiment}

    Two parts:\\
    \begin{enumerate}
        \item Setting the types
        \item Updating
    \end{enumerate}
    \bigskip
    Two treatments:\\
    \begin{enumerate}
        \item Ego
        \item Stereotype
    \end{enumerate}
\end{frame}

\begin{frame}{Set the Types}
    \bigskip
    \begin{itemize}
        \item Quiz: Answer as many questions as you can in 2 minutes\\
        \begin{itemize}
            \item Math, Verbal, Pop-Culture, Science, Us Geography, Sports and Video games\\
        \end{itemize}
        \bigskip

        \item How many questions do you think you answered correctly in each quiz?\\
        \begin{itemize}
            \item 0 to 5 ($\theta_L$)
            \item 6 to 15 ($\theta_M$)
            \item 16 or more ($\theta_H$)
        \end{itemize}
        \item How sure are you about your guess?
        \begin{itemize}
            \item Random guess $\to 1/3$
            \item Another is equally likely $\to 1/2$
            \item Fairly certain $\to 3/4$
            \item Completely sure $\to 1$
        \end{itemize}
    \end{itemize}

\end{frame}


\begin{frame}{Choice and Update}
    ``Effort'' choice and feedback (One topic at a time)\\
    \bigskip
    \begin{itemize}
        \item A success rate is drawn at random (A, B or C)
        \item Choose a gamble: A, B or C (effort)
        \item Receive a sample of 10 signal realizations
    \end{itemize}
    \bigskip
    x 11 per topic

\end{frame}

\begin{frame}{Stereotype condition}

    Observe the characteristics of a participant \\
    \begin{itemize}
        \item Gender 
        \item US National or not \\
    \end{itemize}
    \bigskip
    Answer the same questions about self and other\\

    \bigskip
    Belief updating and effort choice:\\
    \begin{itemize}
        \item  The DGP depends on the $\theta$ the other participant
    \end{itemize}
    \bigskip
    x 11 per topic

\end{frame}

\begin{frame}{Eliciting Beliefs?}
    \begin{itemize}

        \item Track their belief about $\omega$ with their choices\\
        \bigskip
        \item Eliciting beliefs for $\theta$ can incentivize learning in a way that is not consistent with the theory\\
        \bigskip
        
    \end{itemize}

    Allow them to see the probability matrix for only one type 
    \begin{itemize}
        \item Track the matrix they choose to see in each round
    \end{itemize}

\end{frame}


\begin{frame}{Screen}
    \begin{figure}
        \centering
        \includegraphics[scale=.4]{screen1.png}
    \end{figure}
\end{frame}

\begin{frame}{Screen}
    \begin{figure}
        \centering
        \includegraphics[scale=.4]{screen2.png}
    \end{figure}
\end{frame}

\section*{The Data}

\begin{frame}{The Data}
    Subject pool:\\
    \begin{itemize}
        \item Run at the CESS lab in person
        \item 45 subjects in Ego
        \item 33 subjects in Stereotype
    \end{itemize}
    \bigskip
    The Sessions:
    \begin{itemize}
        \item 8 sessions 
        \item About 45 minutes long
        \item Average payment: $\$23$
        \begin{itemize}
            \item $\$10$ show-up fee
            \item $\$ 0.20$ per correct answer
            \item $\$ 0.20$ per success
            \item Paid one topic at random
        \end{itemize}
    \end{itemize}
    
\end{frame}

\section*{Learning}

\begin{frame}{Are they learning $\omega$?}
    \begin{figure}
        \centering
        \includegraphics[scale=.3]{learning_groups.png}
    \end{figure}

\end{frame}

\begin{frame}{Are they learning $\Theta$}
    \begin{figure}
        \centering
        \includegraphics[scale=.33]{last_button_consistency.png}
    \end{figure}
    
\end{frame}

\begin{frame}{Reasons for lack of learning}
\begin{itemize}
    \item Learning traps (self-defeating equilibria)
    \bigskip 
    \item Misattributions
    \bigskip
    \item Others
    \begin{itemize}
        \item Considering the wrong paradigms
        \item Learning is too costly
    \end{itemize}
    
\end{itemize}

\end{frame}

\section*{Learning Traps}




\begin{frame}{Learning when there are traps}
    \begin{figure}
        \centering
        \includegraphics[scale=.5]{learning_trap_presence.png}
    \end{figure}

\end{frame}

\begin{frame}{Are people falling into traps?}
    \begin{figure}
        \centering
        \includegraphics[scale=.5]{trap_learning.png}
    \end{figure}
    \label{traps}
    \action{\hyperlink{categories}{\beamerbutton{paths}}}

\end{frame}


\begin{frame}{Learners, Trapped and Others}
    So far we have seen that:\\
    \begin{itemize}
        \item 40\% of the subjects learn the true state
        \item About 16\% of the subjects fall into self-defeating equilibria
        \item 44\% of the subjects don't learn correctly and don't fall into traps
        \begin{itemize}
            \item From these 60\% were facing parameters for which there were traps
        \end{itemize}
    \end{itemize}
    \bigskip
    How did the learners escape the traps?\\
    \bigskip
    What is the remaining 44\% doing?\\

\end{frame}

\section*{Misattributions}

\begin{frame}{Initial Misspecifications}
    \label{initialhist}
    \begin{figure}
        \centering
        \includegraphics[scale=.6]{misspecification_hist.png}
    \end{figure}
    \action{\hyperlink{certainties}{\beamerbutton{certainties}}}
    \action{\hyperlink{stereotypes}{\beamerbutton{seterotypes}}}

\end{frame}

\begin{frame}{Transition Matrix}
    \begin{figure}
        \centering
        \includegraphics[scale=.6]{misspecification_transitions.png}
    \end{figure}

\end{frame}


\begin{frame}{Good News v. Bad News}
    \label{goodvbad}
    \begin{figure}
        \centering
        \includegraphics[scale=.5]{effort_change_news.png}
    \end{figure}
    \action{\hyperlink{positivevnegative}{\beamerbutton{other}}}

\end{frame}

\begin{frame}{Regression Results}
    \begin{figure}
        \centering
        \includegraphics[scale=.3]{regression_table.png}
    \end{figure}  
\end{frame}

\section*{Stereotypes}

\begin{frame}{Asymmetric Updating in the Stereotype Condition}

    \begin{figure}
        \centering
        \includegraphics[scale=.4]{effort_change_news_treatment.png}
    \end{figure}
    
\end{frame}

\begin{frame}{Do misspecifications persist more often in the Ego condition?}

        \begin{figure}
            \centering
            \includegraphics[scale=.3]{misspecification_transitions_treatment.png}
        \end{figure}
\end{frame}


\begin{frame}{Differences across treatments}
    Very slight differences across treatments\\
    \begin{itemize}
        \item Less stickiness in initial beliefs in Stereotype
        \item Attribution bias in Ego condition
        \item Possible self-censoring in Stereotype
    \end{itemize}


\end{frame}

\section*{Other Explanations}

\begin{frame}{Excessive Switching}
    \begin{figure}
        \centering
        \includegraphics[scale=.5]{switching.png}
    \end{figure}

\end{frame}


\section*{Concluding Remarks}

\begin{frame}{Summary}
Overall:\\
    \begin{itemize}
        \item Traps don't seem to be the main reason for lack of learning
        \item Evidence pointing to misattributions
        \item Ego-relevance seems to play a minor role
    \end{itemize}
    \bigskip
In the presence of traps:\\
    \begin{itemize}
        \item 44\% of subjects learn the true state
        \item About 20\% of the subjects fall into self-defeating equilibria when they exist
        \item 36\% of the subjects don't learn correctly and don't fall into traps
    \end{itemize} 
\bigskip
Stereotypes:\\
    \begin{itemize}
        \item Subjects might be self-censoring their beliefs
        \item Trying to correct initial biases can look like missatribution bias
        \item No confirmation bias
    \end{itemize}
\end{frame}


\begin{frame}{The end}
    \large\textbf{Thank you!}
\end{frame}

\appendix

\begin{frame}{Misspecifications}
    \label{misspecificationsheat}
    \begin{figure}
        \centering
        \includegraphics[scale=.3]{believed_actual_type_heat.png}
    \end{figure}
    \action{\hyperlink{typeheat}{\beamerbutton{Back}}}
\end{frame}

\begin{frame}{Certainties}
    \label{certainties}
    \begin{figure}
        \centering
        \includegraphics[scale=.4]{certainty_by_topic.png}
    \end{figure}

    \action{\hyperlink{initialhist}{\beamerbutton{Back}}}
\end{frame}

\begin{frame}{Misspecification changes by treatment}
    \label{misspecificationsrounds}
    \begin{figure}
        \centering
        \includegraphics[scale=.4]{misspecification_evolution_tratment.png}
    \end{figure}
    \action{\hyperlink{misspecificationsroundstypes}{\beamerbutton{Back}}}
\end{frame}

\begin{frame}{Positive Signals v. Negative Signals}
    \label{positivevnegative}
    \begin{figure}
        \centering
        \includegraphics[scale=.5]{signalvalue_effort_change.png}
    \end{figure}
    \action{\hyperlink{goodvbad}{\beamerbutton{Back}}}

\end{frame}


\begin{frame}{The Stereotypes}
    \label{stereotypes}
    \begin{figure}
        \centering
        \includegraphics[scale=.3]{misspecifications_characteristics_treatment.png}
    \end{figure}

    %add a hyperlink button to the appendix slide with the misspecification_heat
    \action{\hyperlink{misspecificationsheat}{\beamerbutton{types}}} 
    \action{\hyperlink{initialhist}{\beamerbutton{Back}}}

\end{frame}

\begin{frame}{Dogmatic Overconfident: Simulated}
    
    \begin{figure}
        \centering
        \includegraphics[scale=.5]{dogmatic_over_11.png}
        \caption{$\theta^*=\theta_M$, $\hat\theta=\theta_H$, $\omega^*=\omega_M$}
    \end{figure}

    \label{dogmaticpath}
    \action{\hyperlink{dogmatic}{\beamerbutton{Back}}}
\end{frame}


\begin{frame}{Switcher Overconfident: Simulation}
    \begin{figure}
        \centering
        \includegraphics[scale=.5]{switcher_over_11.png}
        \caption{$\theta^*=\theta_M$, $\hat\theta=\theta_H$, $\omega^*=\omega_M$, $\alpha= 1.1$}
    \end{figure}
    \label{switcherpath}
    \action{\hyperlink{switcher}{\beamerbutton{Back}}}
\end{frame}

\begin{frame}{Self-Attribution: Simulation}
    \begin{figure}
        \centering
        \includegraphics[scale=.5]{self-serving_11.png}
    
    \end{figure}
    \label{ss}
    \action{\hyperlink{sspath}{\beamerbutton{Back}}}

\end{frame}


\begin{frame}{Subject categorization}
    \begin{figure}
        \centering
        \includegraphics[scale=.5]{categorized.png}
    \end{figure}
    \label{categories}
    \action{\hyperlink{traps}{\beamerbutton{Back}}}
    
\end{frame}

\end{document}
