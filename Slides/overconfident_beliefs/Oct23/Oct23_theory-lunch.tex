\documentclass[aspectratio=169]{beamer}
\usetheme{metropolis}
\usepackage{geometry}
\usepackage{amsmath}
\usepackage{graphicx}
\usepackage{amsfonts}
\usepackage{amssymb}
\usepackage{setspace}
\usepackage{theorem}
\usepackage{natbib}
\usepackage{mathtools}
\usepackage{cite}
\usepackage{natbib}
\usepackage{setspace}
\usepackage[utf8]{inputenc}
\usepackage[english]{babel}
\usepackage{xcolor}
\usepackage{array}
\usepackage{caption}
\usepackage{graphicx}
\usepackage{siunitx}
\usepackage[normalem]{ulem}
\usepackage{colortbl}
\usepackage{multirow}
\usepackage{hhline}
\usepackage{calc}
\usepackage{tabularx}
\usepackage{threeparttable}
\usepackage{wrapfig}
\usepackage{adjustbox}
\usepackage{hyperref}
\usepackage{tikz}
\usepackage[table]{xcolor}

\title{Learning with Misspecified Models:\\  the case of overconfidence}
\author{Jimena Galindo}

  
\begin{document}

\frame{\titlepage}

\begin{frame}{Overconfidence is Costly}
    \textbf{OVERCONFIDENCE}: Belief that my type is higher than it truly is (``overestimation'' as in Moore and Healy (2008))\\
    \bigskip
    \pause
    It seems to be persistent in various settings. 
    \begin{itemize} 
        \item Excess entry of entrepreneurs (Camerer and Lovallo, 1999)
        \item Suboptimal genetic testing and healthcare (Oster et al. 2013)
        \item  Workers overestimate their productivity (Hoffman and Burks, 2020)
    \end{itemize}
    \bigskip
    \alert{Ultimately it leads to sub-optimal choices}
    
\end{frame}

\begin{frame}{Models of Learning}
    Some of the features that theory has incorporated to explain overconfidence are:
    \begin{itemize}
        \item Dogmatism
        \item Paradigm shifts
        \item Motivated beliefs
        \item Myopic optimiztion
    \end{itemize}
\end{frame}

\begin{frame}{Four Theories of Misspecified Learning}
    \begin{enumerate}
        \item Self-defeating equilibrium (Heidhues et al. (2018)): 
        \begin{itemize}
            \item Bayesian on $\omega$
            \item Dogmatic about $\theta$
            
        \end{itemize}
        \item Bayesian Likelihood Ratio test (Schwarstein and Sunderam (2021), Ba, (2022 JMP)) :
        \begin{itemize}
            \item Bayesian on $\omega$ 
            \item Hypothesis testing on $\theta$
        \end{itemize}
        \item Motivated Beliefs or Self-Attribution Bias (Benjamin, 2019): 
        \begin{itemize}
            \item Errors in probabilistic reasoning and judgment biases
        \end{itemize}
    \end{enumerate}
    
\end{frame}


\begin{frame}{An Example}
    A student has \textbf{unknown intrinsic ability} $\theta^*$ and chooses a level of effort $e\geq 0$. \\
    \bigskip
    Effort and ability are transformed into a noisy output at an exogenous and \textbf{unknown rate} $\omega.$\\
    \bigskip 
    An overconfident student believes he is of type $\hat\theta>\theta^*$\\
    \bigskip
    
    And wants to maximize utility\\
        $$y = (\theta^* + e)\omega-\frac{1}{2}e^2 +\varepsilon$$\\
    \pause
    \bigskip
   {\textbf{Regardless of their own type, they should choose $e^*(\omega)=\omega$\\}}
\end{frame}

\begin{frame}{Learning is Possible}
    This exercise is repeated for $t=0, 1, ...$
        $$y_t = (\theta^* + e_t)\omega-\frac{1}{2}e_t^2 +\varepsilon_t$$
    
    Note that both parameters are identified in this setting:\\
    \bigskip
    
    \begin{itemize}
        \item Choosing $\hat{e}$ and $\hat{e}+1$ over multiple periods allows identification of $\omega$\\
        \bigskip
        \item Once $\omega$ is known, $\theta$ can be backed out\\
     \end{itemize}
    \bigskip
    How come people don't learn their true type and don't choose the optimal effort?
\end{frame}

\begin{frame}{Mental Models}

A prior belief over parameters/states/types and an updating procedure\\
\bigskip
\begin{itemize}
    \item Bayesian 
    \item Dogmatic 
    \item Motivated Beliefs/Self-attribution
\end{itemize}
\end{frame}


\begin{frame}{Research Questions}
    To what extent do the different theories explain observed behavior?\\
    \begin{itemize}
        \item Do we observe heterogeneity in the use of mental models?
    \end{itemize}
    \bigskip
    Is ego-relevance of the type a key feature for the misspecification?
    \begin{itemize}
        \item Can the same theories be used to explain the prevalence of stereotypes?
    \end{itemize}
\end{frame}

\begin{frame}{Road-map}
    \begin{enumerate}
        \item Three Theories of Overconfidence\\
        \bigskip
        \item Mechanisms and Predictions\\
        \bigskip
        \item Unifying Framework\\
        \bigskip
        \item Experimental Design\\
        \bigskip
        \item Results (coming soon)
    \end{enumerate}
\end{frame}

\begin{frame}{The Theories}
Settings with two or more unknowns allow for different explanations of the bias:\\
\bigskip
\begin{enumerate}
\item Self-defeating equilibrium (Heidhues et al., 2018): 
    \begin{itemize}
        \item Bayesian on $\omega$
        \item Dogmatic about $\theta$
        
    \end{itemize}
    \bigskip
    \item Bayesian Likelihood Ratio test (Ba, 2022 JMP):
    \begin{itemize}
        \item Bayesian on $\omega$ 
        \item Hypothesis testing on $\theta$
    \end{itemize}
    \bigskip
    \item Self-Serving Attribution Bias with two unknowns (Brunnermeier and Parker, 2005; Coutts et al. 2022wp): 
    \begin{itemize}
        \item Good news are attributed to high $\theta$ bad news are attributed to low $\omega$
    \end{itemize}
    
    
    
\end{enumerate}
\end{frame} 

\begin{frame}{Theory 1: Dogmatic Modelers (HKS)}
    \Large\textbf{Unrealistic Expectations and Misguided Learning \\}
    (Heidhues, Köszegi, and Strack, 2018)
\end{frame}

\begin{frame}{The Setting}

The student's true ability is $\theta^*$, they believe with certainty that it is $\hat\theta>\theta^*.$ \\
\bigskip
The rate $\omega$ is drawn from density $g_0$ with $\omega^*=E_{g_0}(\omega).$\\

\bigskip
At $t=0$, the student has the prior $g_0.$\\
\bigskip
They correctly choose $e_0 = \omega^*.$\\
\bigskip
\pause
\textbf{Suppose they don't update their beliefs or their choice for a number of periods.}
\end{frame}

\begin{frame}{Updating the Beliefs}

For their chosen effort $\omega^*$, they observe an average output of 
$$ y_0=(\theta^* + \omega^*)\omega^*-\frac{1}{2}(\omega^*)^2 $$

But were expecting
$$ (\hat\theta + \omega^*)\omega^*-\frac{1}{2}(\omega^*)^2 > y_0$$\\
\bigskip
\pause
So they conclude that $\omega_1$ must be such that:
$$(\hat \theta + \omega^*)\omega_1-\frac{1}{2}(\omega^*)^2 = (\theta^* + \omega^*)\omega^*-\frac{1}{2}(\omega^*)^2 $$\\
\bigskip 
Which gives $\omega_1 = \frac{(\theta^* + \omega^*)\omega^*}{(\hat \theta + \omega^*)}<\omega^*$
    
\end{frame}

\begin{frame}{Updating the Beliefs}
    Updating choices every period (myopically) the belief will drift even further:\\
    \bigskip
    A lower choice of $e$ still gives a lower output than expected. \\
    \bigskip
    So $\omega_{t+1}$ must be lower than they believed in period $t$.\\
    \bigskip
    \textbf{Prediction:} convergence to a self-confirming equilibrium with $\omega_\infty<\omega_1<\omega^*.$\\
    \bigskip
    The result is symmetric for underconfident subjects.
    
\end{frame}


\begin{frame}{Theory 2: Switchers}
    \Large\textbf{ Robust Misspecified Models and Paradigm Shifts \\}
    (Ba, 2022 JMP)
\end{frame}


\begin{frame}{The Setting}
    Same as HKS but with finite $\Omega$ and finite $A$\\
    \bigskip
    Now the entrepreneur is willing to switch to an alternative level of ability $\theta'$ (assume $\theta' = \theta^*$).\\
    \bigskip
    Instead of updating $P[\theta]$ every period, they perform a Bayesian hypothesis test:\\
    \bigskip
    Adopt model $\theta'$ at time t iff \\
    $$\frac{\ell_t (\theta')}{\ell_t(\hat{\theta})}>\alpha\geq1$$

    Where $$\ell_t(\theta) := \sum_{\omega}g_0(\omega)\prod_{\tau=0}^{t-1}\pi^\theta(y_\tau|a_\tau, \omega)$$
  
\end{frame}

\begin{frame}{Results}

    \textbf{Prediction:} Misspecified agents escape the trap as long as their prior is not too ``tight'' around a self-confirming equilibrium.
  
\end{frame}

\begin{frame}{Theory 3: Motivated Beliefs}
    
    \Large\textbf{Errors in probabilistic reasoning and judgment biases}\\
    (Benjamin, 2019)\\
    
\end{frame}

\begin{frame}{The Setting}
    Fixed effort $e$, $\theta \in \{\theta_H, \theta_L\}$ and $\omega \in \{\omega_H, \omega_L\}$ generate binary signals (\textbf{s}/\textbf{f})\\
    \bigskip
    After a signal realization $m$, the agent updates their belief about $\theta$ with distortions $c_m^\theta$ and $c_m^\omega$, so that:\\

        $$\frac{p_{t+1}[\theta_H]}{p_{t+1}[\theta_L]} = \left(\frac{p[\text{m}|\theta_H]}{p[\text{m}|\theta_L]}\right)^{c_m^\theta}\frac{p_t[\theta_H]}{p_t[\theta_L]}$$
    
    and
        $$\frac{p_{t+1}[\omega_H]}{p_{t+1}[\omega_L]} = \left(\frac{p[\text{m}|\omega_H]}{p[\text{m}|\omega_L]}\right)^{c_m^\omega}\frac{p_t[\omega_H]}{p_t[\omega_L]}$$
    
    \bigskip
    The agent suffers from self-attribution bias if $c_s^\theta >c_f^\theta$ and $c_s^\omega < c_f^\omega$.
\end{frame}

\begin{frame}{Predictions}
    \textbf{Prediction:} Even unbiased agents will overweight $\theta_H$ after a success and end up being biased.\\
    \bigskip
    When $c^\theta = c^\omega = 1$, the updating procedure coincides with the unbiased Bayesian.\\
    
    \bigskip
    The framework does not allow direct comparisons with the other two theories.\\
    
    
\end{frame}

\begin{frame}{A Unifying Framework}

Finite type space: $\theta \in \{\theta_H, \theta_M, \theta_L\}$\\
\bigskip
Finite state space: $\omega \in \{\omega_H, \omega_M, \omega_L\}$
with $p(\omega_k)=1/3$ \\
\bigskip
Finite action space: $e \in \{e_H, e_M, e_L\}$\\
\bigskip
Binary signal: Success/Failure with $P\left[Success|e, \omega, \theta\right]$ satisfying the assumptions of HKS


\end{frame}

\begin{frame}{The Data Generating Process}
The probability of success is given by:\\
\bigskip
\centering
\begin{tabular}{ c|c|c|c|}
  
  \multicolumn{1}{c}{} & \multicolumn{1}{c}{$\omega_H$} & \multicolumn{1}{c}{$\omega_M$} & \multicolumn{1}{c}{$\omega_L$}\\
  \cline{2-4}
  $e_H$ & 50 & 20 & 2 \\
  \cline{2-4}
  $e_M$ & 45 & 30 & 7 \\
  \cline{2-4}
  $e_L$ & 40 & 25 & 20 \\
  \cline{2-4}
  \multicolumn{1}{c}{} & \multicolumn{1}{c}{} & \multicolumn{1}{c}{$\theta_L$} & \multicolumn{1}{c}{}\\
\end{tabular}
\hspace{.3cm} % adjust this value to set the space between tables
\begin{tabular}{ c|c|c|c|}
  
  \multicolumn{1}{c}{} & \multicolumn{1}{c}{$\omega_H$} & \multicolumn{1}{c}{$\omega_M$} & \multicolumn{1}{c}{$\omega_L$}\\
  \cline{2-4}
  $e_H$ & 80 & 50 & 5 \\
  \cline{2-4}
  $e_M$ & 69 & 65 & 30 \\
  \cline{2-4}
  $e_L$ & 65 & 45 & 40 \\
  \cline{2-4}
  \multicolumn{1}{c}{} & \multicolumn{1}{c}{} & \multicolumn{1}{c}{$\theta_M$} & \multicolumn{1}{c}{}\\
\end{tabular}
\hspace{.3cm} % adjust this value to set the space between tables
\begin{tabular}{ c|c|c|c|}
  
  \multicolumn{1}{c}{} & \multicolumn{1}{c}{$\omega_H$} & \multicolumn{1}{c}{$\omega_M$} & \multicolumn{1}{c}{$\omega_L$}\\
  \cline{2-4}
  $e_H$ & 98 & 65 & 25 \\
  \cline{2-4}
  $e_M$ & 80 & 69 & 35 \\
  \cline{2-4}
  $e_L$ & 75 & 55 & 45 \\
  \cline{2-4}
  \multicolumn{1}{c}{} & \multicolumn{1}{c}{} & \multicolumn{1}{c}{$\theta_H$} & \multicolumn{1}{c}{}\\
\end{tabular}


    
\end{frame}

\begin{frame}{The Data Generating Process}
    \centering
\begin{tabular}{ c|c|c|c|}
  
  \multicolumn{1}{c}{} & \multicolumn{1}{c}{$\omega_H$} & \multicolumn{1}{c}{$\omega_M$} & \multicolumn{1}{c}{$\omega_L$}\\
  \cline{2-4}
  $e_H$ & \cellcolor{blue!25}50 & 20 & 2 \\
  \cline{2-4}
  $e_M$ & 45 & \cellcolor{blue!25}30 & 7 \\
  \cline{2-4}
  $e_L$ & 40 & 25 & \cellcolor{blue!25}20 \\
  \cline{2-4}
  \multicolumn{1}{c}{} & \multicolumn{1}{c}{} & \multicolumn{1}{c}{$\theta_L$} & \multicolumn{1}{c}{}\\
\end{tabular}
\hspace{.3cm} % adjust this value to set the space between tables
\begin{tabular}{ c|c|c|c|}
  
  \multicolumn{1}{c}{} & \multicolumn{1}{c}{$\omega_H$} & \multicolumn{1}{c}{$\omega_M$} & \multicolumn{1}{c}{$\omega_L$}\\
  \cline{2-4}
  $e_H$ & \cellcolor{blue!25}80 & 50 & 5 \\
  \cline{2-4}
  $e_M$ & 69 & \cellcolor{blue!25}65 & 30 \\
  \cline{2-4}
  $e_L$ & 65 & 45 & \cellcolor{blue!25}40 \\
  \cline{2-4}
  \multicolumn{1}{c}{} & \multicolumn{1}{c}{} & \multicolumn{1}{c}{$\theta_M$} & \multicolumn{1}{c}{}\\
\end{tabular}
\hspace{.3cm} % adjust this value to set the space between tables
\begin{tabular}{ c|c|c|c|}
  
  \multicolumn{1}{c}{} & \multicolumn{1}{c}{$\omega_H$} & \multicolumn{1}{c}{$\omega_M$} & \multicolumn{1}{c}{$\omega_L$}\\
  \cline{2-4}
  $e_H$ & \cellcolor{blue!25}98 & 65 & 25 \\
  \cline{2-4}
  $e_M$ & 80 & \cellcolor{blue!25}69 & 35 \\
  \cline{2-4}
  $e_L$ & 75 & 55 & \cellcolor{blue!25}45 \\
  \cline{2-4}
  \multicolumn{1}{c}{} & \multicolumn{1}{c}{} & \multicolumn{1}{c}{$\theta_H$} & \multicolumn{1}{c}{}\\
\end{tabular}
\end{frame}



\begin{frame}{The Data Generating Process}
\begin{center}
\begin{tikzpicture}
  \draw[<-] (1,0) -- (3,0);
  \draw[<-] (5,0) -- (7,0);
  \draw[<-] (9,0) -- (11,0);
\end{tikzpicture}
\end{center}
\\
\bigskip
\centering
\begin{tabular}{ c|c|c|c|}
  
  \multicolumn{1}{c}{} & \multicolumn{1}{c}{$\omega_H$} & \multicolumn{1}{c}{$\omega_M$} & \multicolumn{1}{c}{$\omega_L$}\\
  \cline{2-4}
  $e_H$ & \cellcolor{blue!25}50 & 20 & 2 \\
  \cline{2-4}
  $e_M$ & 45 & \cellcolor{blue!25}30 & 7 \\
  \cline{2-4}
  $e_L$ & 40 & 25 & \cellcolor{blue!25}20 \\
  \cline{2-4}
  \multicolumn{1}{c}{} & \multicolumn{1}{c}{} & \multicolumn{1}{c}{$\theta_L$} & \multicolumn{1}{c}{}\\
\end{tabular}
\hspace{.3cm} % adjust this value to set the space between tables
\begin{tabular}{ c|c|c|c|}
  
  \multicolumn{1}{c}{} & \multicolumn{1}{c}{$\omega_H$} & \multicolumn{1}{c}{$\omega_M$} & \multicolumn{1}{c}{$\omega_L$}\\
  \cline{2-4}
  $e_H$ & \cellcolor{blue!25}80 & 50 & 5 \\
  \cline{2-4}
  $e_M$ & 69 & \cellcolor{blue!25}65 & 30 \\
  \cline{2-4}
  $e_L$ & 65 & 45 & \cellcolor{blue!25}40 \\
  \cline{2-4}
  \multicolumn{1}{c}{} & \multicolumn{1}{c}{} & \multicolumn{1}{c}{$\theta_M$} & \multicolumn{1}{c}{}\\
\end{tabular}
\hspace{.3cm} % adjust this value to set the space between tables
\begin{tabular}{ c|c|c|c|}
  
  \multicolumn{1}{c}{} & \multicolumn{1}{c}{$\omega_H$} & \multicolumn{1}{c}{$\omega_M$} & \multicolumn{1}{c}{$\omega_L$}\\
  \cline{2-4}
  $e_H$ & \cellcolor{blue!25}98 & 65 & 25 \\
  \cline{2-4}
  $e_M$ & 80 & \cellcolor{blue!25}69 & 35 \\
  \cline{2-4}
  $e_L$ & 75 & 55 & \cellcolor{blue!25}45 \\
  \cline{2-4}
  \multicolumn{1}{c}{} & \multicolumn{1}{c}{} & \multicolumn{1}{c}{$\theta_H$} & \multicolumn{1}{c}{}\\
\end{tabular}

\begin{center}
    \resizebox{0.6\linewidth}{!}{\vector(1,0){300}}
  \end{center}
    
\end{frame}

\begin{frame}{A Self-Confirming Equilibrium}

\centering
\begin{tabular}{ c|c|c|c|}
  
  \multicolumn{1}{c}{} & \multicolumn{1}{c}{$\omega_H$} & \multicolumn{1}{c}{$\omega_M$} & \multicolumn{1}{c}{$\omega_L$}\\
  \cline{2-4}
  $e_H$ & \cellcolor[HTML]{b84f79}50 & 20 & 2 \\
  \cline{2-4}
  $e_M$ & 45 & 30 & 7 \\
  \cline{2-4}
  $e_L$ & 40 & 25 & 20 \\
  \cline{2-4}
  \multicolumn{1}{c}{} & \multicolumn{1}{c}{} & \multicolumn{1}{c}{$\theta_L$} & \multicolumn{1}{c}{}\\
\end{tabular}
\hspace{.3cm} % adjust this value to set the space between tables
\begin{tabular}{ c|c|c|c|}
  
  \multicolumn{1}{c}{} & \multicolumn{1}{c}{$\omega_H$} & \multicolumn{1}{c}{$\omega_M$} & \multicolumn{1}{c}{$\omega_L$}\\
  \cline{2-4}
  $e_H$ & 80 & \cellcolor[HTML]{b84f79}50 & 5 \\
  \cline{2-4}
  $e_M$ & 69 &\cellcolor[HTML]{f09ebe}65 & 30 \\
  \cline{2-4}
  $e_L$ & 65 & \cellcolor[HTML]{f09ebe}45 & 40 \\
  \cline{2-4}
  \multicolumn{1}{c}{} & \multicolumn{1}{c}{} & \multicolumn{1}{c}{$\theta_M$} & \multicolumn{1}{c}{}\\
\end{tabular}
\hspace{.3cm} % adjust this value to set the space between tables
\begin{tabular}{ c|c|c|c|}
  
  \multicolumn{1}{c}{} & \multicolumn{1}{c}{$\omega_H$} & \multicolumn{1}{c}{$\omega_M$} & \multicolumn{1}{c}{$\omega_L$}\\
  \cline{2-4}
  $e_H$ & 98 & 65 & 25 \\
  \cline{2-4}
  $e_M$ & 80 & 69 & 35 \\
  \cline{2-4}
  $e_L$ & 75 & 55 & 45 \\
  \cline{2-4}
  \multicolumn{1}{c}{} & \multicolumn{1}{c}{} & \multicolumn{1}{c}{$\theta_H$} & \multicolumn{1}{c}{}\\
\end{tabular}
\end{frame}


\begin{frame}{The Self-Confirming Equilibria}

\centering
\begin{tabular}{ c|c|c|c|}
  
  \multicolumn{1}{c}{} & \multicolumn{1}{c}{$\omega_H$} & \multicolumn{1}{c}{$\omega_M$} & \multicolumn{1}{c}{$\omega_L$}\\
  \cline{2-4}
  $e_H$ & \cellcolor[HTML]{b84f79}50 & 20 & 2 \\
  \cline{2-4}
  $e_M$ & 45 & \cellcolor[HTML]{5f94b8}30 & 7 \\
  \cline{2-4}
  $e_L$ & \cellcolor[HTML]{69a35b}40 & 25 & 20 \\
  \cline{2-4}
  \multicolumn{1}{c}{} & \multicolumn{1}{c}{} & \multicolumn{1}{c}{$\theta_L$} & \multicolumn{1}{c}{}\\
\end{tabular}
\hspace{.3cm} % adjust this value to set the space between tables
\begin{tabular}{ c|c|c|c|}
  
  \multicolumn{1}{c}{} & \multicolumn{1}{c}{$\omega_H$} & \multicolumn{1}{c}{$\omega_M$} & \multicolumn{1}{c}{$\omega_L$}\\
  \cline{2-4}
  $e_H$ & 80 & \cellcolor[HTML]{b84f79}50 & 5 \\
  \cline{2-4}
  $e_M$ & \cellcolor[HTML]{fab143}69 & 65 & \cellcolor[HTML]{5f94b8}30 \\
  \cline{2-4}
  $e_L$ & 65 & \cellcolor[HTML]{9662f0}45 & \cellcolor[HTML]{69a35b}40 \\
  \cline{2-4}
  \multicolumn{1}{c}{} & \multicolumn{1}{c}{} & \multicolumn{1}{c}{$\theta_M$} & \multicolumn{1}{c}{}\\
\end{tabular}
\hspace{.3cm} % adjust this value to set the space between tables
\begin{tabular}{ c|c|c|c|}
  
  \multicolumn{1}{c}{} & \multicolumn{1}{c}{$\omega_H$} & \multicolumn{1}{c}{$\omega_M$} & \multicolumn{1}{c}{$\omega_L$}\\
  \cline{2-4}
  $e_H$ & 98 & 65 & 25 \\
  \cline{2-4}
  $e_M$ & 80 & \cellcolor[HTML]{fab143}69 & 35 \\
  \cline{2-4}
  $e_L$ & 75 & 55 & \cellcolor[HTML]{9662f0}45 \\
  \cline{2-4}
  \multicolumn{1}{c}{} & \multicolumn{1}{c}{} & \multicolumn{1}{c}{$\theta_H$} & \multicolumn{1}{c}{}\\
\end{tabular}
\end{frame}

\begin{frame}{An Example}
\begin{itemize}
    \item True type is $\theta_M$ \\
    \bigskip
    \item True exchange rate is $\omega_M$ $\rightarrow$ The entrepreneur believes it is uniformly distributed\\
    \end{itemize}

    \centering
\begin{tabular}{ c|c|c|c|}
  
  \multicolumn{1}{c}{} & \multicolumn{1}{c}{$\omega_H$} & \multicolumn{1}{c}{$\omega_M$} & \multicolumn{1}{c}{$\omega_L$}\\
  \cline{2-4}
  $e_H$ & 50 & 20 & 2 \\
  \cline{2-4}
  $e_M$ & 45 & 30 & 7 \\
  \cline{2-4}
  $e_L$ & 40 & 25 & 20 \\
  \cline{2-4}
  \multicolumn{1}{c}{} & \multicolumn{1}{c}{} & \multicolumn{1}{c}{$\theta_L$} & \multicolumn{1}{c}{}\\
\end{tabular}
\hspace{.3cm} % adjust this value to set the space between tables
\begin{tabular}{ c|c|c|c|}
  
  \multicolumn{1}{c}{} & \multicolumn{1}{c}{$\omega_H$} & \multicolumn{1}{c}{$\omega_M$} & \multicolumn{1}{c}{$\omega_L$}\\
  \cline{2-4}
  $e_H$ & 80 & \cellcolor{blue!25}50 & 5 \\
  \cline{2-4}
  $e_M$ & 69 & \cellcolor{blue!25}65 & 30 \\
  \cline{2-4}
  $e_L$ & 65 & \cellcolor{blue!25}45 & 40 \\
  \cline{2-4}
  \multicolumn{1}{c}{} & \multicolumn{1}{c}{} & \multicolumn{1}{c}{$\theta_M$} & \multicolumn{1}{c}{}\\
\end{tabular}
\hspace{.3cm} % adjust this value to set the space between tables
\begin{tabular}{ c|c|c|c|}
  
  \multicolumn{1}{c}{} & \multicolumn{1}{c}{$\omega_H$} & \multicolumn{1}{c}{$\omega_M$} & \multicolumn{1}{c}{$\omega_L$}\\
  \cline{2-4}
  $e_H$ & 98 & 65 & 25 \\
  \cline{2-4}
  $e_M$ & 80 & 69 & 35 \\
  \cline{2-4}
  $e_L$ & 75 & 55 & 45 \\
  \cline{2-4}
  \multicolumn{1}{c}{} & \multicolumn{1}{c}{} & \multicolumn{1}{c}{$\theta_H$} & \multicolumn{1}{c}{}\\
\end{tabular}
    
\end{frame}

\begin{frame}{Example: Dogmatic Modeler}
\begin{itemize}
    \item Theory 1: for a student who believes he is $\theta_H$
    \end{itemize}
    \bigskip
    \begin{enumerate}
        \item Chooses $e_H$ and is disappointed $\rightarrow$ adjust belief about $\omega$ downward\\
        \bigskip
        \item Eventually chooses $e_M$ and is disappointed as well $\rightarrow$ adjust belief about $\omega$\\
        \bigskip
        \item Eventually chooses $e_L$ and falls into a self-confirming equilibrium
    \end{enumerate}
    
    
\end{frame}

\begin{frame}{Dogmatic Overconfident}
    \begin{figure}
        \centering
        \includegraphics[scale=.5]{figures2/dogmatic_over_11.png}
        \caption{$\theta^*=\theta_M$, $\hat\theta=\theta_H$, $\omega^*=\omega_M$}
    \end{figure}
\end{frame}


\begin{frame}{Example: Likelihood Testing}
\begin{itemize}
    \item Theory 2: for the same initial belief
    \item Keeping track of the likelihood of each $\theta$
    \end{itemize}
    \bigskip
    \begin{enumerate}
        \item Chooses $e_H$ and is disappointed $\rightarrow$ adjust belief about $\omega$ downward\\
        \bigskip
        \item Eventually chooses $e_M$ and is disappointed as well $\rightarrow$ adjust belief about $\omega$\\
        \bigskip
        \item Eventually chooses $e_L$ and falls into a self-confirming equilibrium\\
        \bigskip
        \item At some point, the likelihood of $\theta_M$ becomes much larger than that of $\theta_H$ and the agent updates their belief
    \end{enumerate}
    
    
\end{frame}

\begin{frame}{Switcher Overconfident}
    \begin{figure}
        \centering
        \includegraphics[scale=.5]{figures2/switcher_over_11.png}
        \caption{$\theta^*=\theta_M$, $\hat\theta=\theta_H$, $\omega^*=\omega_M$, $\alpha= 1.1$}
    \end{figure}
\end{frame}


\begin{frame}{Example: Self-Serving Beliefs}
\begin{itemize}
    \item Theory 3: Start with a diffused prior over $\theta$
    \end{itemize}
    \bigskip
    \begin{enumerate}
        \item Chooses $e$ that maximizes utility according to priors
        \bigskip
        \item Success $\rightarrow$ overweight $\theta_H$ and underweight $\omega_H$\\
        \item Failure $\rightarrow$ overweight $\omega_L$ underweight $\theta_L$\\
        \bigskip
        \item Belief on $\omega$ deteriorates a lot after failure streaks
        \item Belief on $\theta$ increases a lot after success streaks\\
    \end{enumerate}
    
    
\end{frame}

\begin{frame}{Self-Serving Bias}
    \begin{figure}
        \centering
        \includegraphics[scale=.5]{figures2/self-serving_11.png}
    
    \end{figure}
\end{frame}


\begin{frame}{The Simulation}
\label{Figure2}
    \begin{figure}
    \centering
    \includegraphics[scale=0.2]{figures2/all_11.png}
\end{figure}     



\end{frame}


\begin{frame}{The Experiment:}

Part 1: Set Types\\
\bigskip
\begin{itemize}
    \item Quiz: Answer as many questions as you can in 2 minutes:\\
    \begin{itemize}
        \item Math, Verbal, Pop-Culture, Science, Us Geography, Sports and Video games\\
    \end{itemize}
    \bigskip
    \item How many questions do you think you answered correctly in each quiz?\\
    \begin{itemize}
        \item[o] Bin1, Bin2, Bin3
    \end{itemize}
\end{itemize}

\end{frame}


\begin{frame}{The Experiment: Ego-relevant condition}
Belief updating and effort choice (One topic at a time)\\
\bigskip
\begin{itemize}
    \item Choose an effort
    \item Receive a sample of 10 signal realizations
\end{itemize}
\bigskip
11 rounds per topic

\end{frame}

\begin{frame}{Eliciting Beliefs?}
\begin{itemize}
    \item $E[\omega]$ is revealed by their choice of effort\\
    \bigskip
    \item Eliciting beliefs for $\theta$ can incentivize learning in a way that is not consistent with the model\\
    \bigskip
\end{itemize}

Allow them to see the success rate matrix for only one type. 
\begin{itemize}
    \item Track the matrices they choose to see in each round
\end{itemize}



\end{frame}

\begin{frame}{The Experiment: Stereotype condition}

Observe the characteristics of a participant (Gender, US National or not). \\
\begin{itemize}
    \item ``What score do you think this participant got in the (topic) quiz?'' \\
    \item Bin1, Bin2, Bin3 \\
\end{itemize}
\bigskip
Belief updating and effort choice\\

\begin{itemize}
    \item Choose an effort
    \item Receive a signal realization
    \begin{itemize}
        \item[o] The DGP is that of the observed participant
    \end{itemize}
\end{itemize}
\bigskip
11 rounds (per topic/participant)

\end{frame}

\begin{frame}{Screen}
    \begin{figure}
        \centering
        \includegraphics[scale=.4]{figures2/screen1.png}
    \end{figure}
\end{frame}

\begin{frame}{Screen}
    \begin{figure}
        \centering
        \includegraphics[scale=.4]{figures2/screen2.png}
    \end{figure}
\end{frame}


\begin{frame}{Conclusion}
What I hope to get from this design:\\
\bigskip
    \begin{itemize}
        \item A classification of subjects into one of the models based on their behavior\\
        \bigskip
        \item If subjects are switchers: what is the switching threshold $\alpha$\\
        \bigskip
        \item Insight into the role of ego-relevant parameters in belief misspecification
    \end{itemize}
\end{frame}

\begin{frame}{The end}
    \large\textbf{Thank you!}
\end{frame}





\end{document}





\end{document}